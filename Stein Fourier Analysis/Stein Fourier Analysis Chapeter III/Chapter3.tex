\documentclass[12pt, a4paper, oneside]{ctexart}
\usepackage{amsmath, amsthm, amssymb, bm, color, framed, graphicx, hyperref, mathrsfs}

\title{\textbf{Fourier Analysis Chapter 3}}
\author{张浩然}
\date{\today}
\linespread{1.5}
\definecolor{shadecolor}{RGB}{241, 241, 255}
\newcounter{problemname}
\newenvironment{problem}{\begin{shaded}\stepcounter{problemname}\par\noindent\textbf{题目\arabic{problemname}. }}{\end{shaded}\par}
\newenvironment{solution}{\par\noindent\textbf{解答. }}{\par}
\newenvironment{note}{\par\noindent\textbf{题目\arabic{problemname}的注记. }}{\par}

\begin{document}

\maketitle


\begin{problem}
Exercise 1:
\par
证明: $d$有限时,$\mathbb{R}^d$和$\mathbb{C}^d$是完备的.
\end{problem}

\begin{solution}
证明:
\par
只需要证明Cauchy列收敛,并且此极限在空间中.
\par
我们先研究$\mathbb{R}^{d}$,取一般的$2-$范数$\|\cdot\|$.
\par
取$\mathbb{R}^d$中柯西列$\{a_n\}$,那么任取$\varepsilon>0$,存在$N\in \mathbb{N}$,
使得$m,n>N$时,
$$
\|a_m-a_n\|<\varepsilon
$$
\par
则$$
|a_m^i-a_n^i|\leqslant \|a_m-a_n\|<\varepsilon, i=1,2,\cdots,d
$$
\par
则$\{a_n^i\}$是$\mathbb{R}$上Cauchy列,则其收敛到一个实数$a^i\in\mathbb{R}$
\par
$$
\lim_{n\to \infty}a_n=(a^1,a^2,\cdots,a^d)\in \mathbb{R}^d
$$
\par
则$\mathbb{R}^d$是完备的.
\par
将上述论证相关部分换成如下对应:
\par
$w\in \mathbb{R}^d$,
$\|w\|=\sqrt{|w^1|^2+\cdots+|w^d|^2}$
\par
取$\mathbb{C}^d$中Cauchy列,利用$\mathbb{C}$完备性同理.
\end{solution}


\begin{problem}
Exercise 2:
\par
证明: $l^2(\mathbb{Z})$是完备的.
\end{problem}

\begin{solution}
证明:
\par
取$l^2(\mathbb{Z})$中
的Cauchy列$\{a_{k,n}\}_{k\in \mathbb{Z}_{\geqslant 1},n \in \mathbb{Z}}$
\par
设$A_k=a_{k,n}$,取$l^2$范数,则:任意$\varepsilon>0$,存在$N\in \mathbb{N}$,
使得当$k,k'>N$时,有
$$
\|A_k-A_{k'}\|<\varepsilon
$$
\par
固定$n$,那么$$
|a_{k,n}-a_{k',n}|\leqslant\|A_k-A_{k'}\|<\varepsilon
$$
\par
那么$a_{k,n}$是$\mathbb{C}$的柯西列,则其收敛到复数$a_n\in \mathbb{C}$
\par
则$$
\lim_{n \to \infty}A_k=B=(\cdots, a_{-1},a_0,a_1,\cdots)
$$
\par
让$k'\to \infty$,有:
$$
\|A_k-B\|<\varepsilon, k>N
$$
\par
则存在$M>0$,使得$\|A_k-B\|\leqslant M, \forall k \in \mathbb{Z}_{\geqslant 1}$.
\par
那么,$$
\|B\|\leqslant \|A_k\|+\|A_k-B\|<+\infty
$$
\par
$B \in l^2(\mathbb{Z})$,这就证明了$l^2(\mathbb{Z})$是完备的,进一步是Hilbert空间.
\end{solution}


\begin{problem}
    Exercise 5:
    \par
    已知$$
    f(\theta)=
    \begin{cases}
     0,   &\theta=0;\\
     \log\left(\dfrac{1}{\theta}\right), &0<\theta\leqslant 2\pi.
    \end{cases}
    $$
    \par
    定义函数列:
    \par
    $$
    f_n(\theta)=
    \begin{cases}
     0,   &0\leqslant \theta \leqslant \dfrac{1}{n};\\
     f(\theta), &\dfrac{1}{n}<\theta\leqslant 2\pi.
    \end{cases}
    $$
    \par
    证明:$\{f_n\}$是$\mathcal{R}$(Riemann可积函数空间)中的Cauchy列,但是$f\notin \mathcal{R}$.
\end{problem}

\begin{solution}
\par
证明:
\par
不妨设$m>n,m,n\in \mathbb{Z}_{\geqslant 1}$,那么:$\dfrac{1}{n}>\dfrac{1}{m}$
$$
\begin{aligned}
 &\dfrac{1}{2\pi}\int_{0}^{2\pi}|f_m(\theta)-f_n(\theta)|^2\mathrm{d}\theta\\
=&\dfrac{1}{2\pi}\int_{\frac{1}{m}}^{\frac{1}{n}}\log^2\left(\dfrac{1}{\theta}\right)\mathrm{d}\theta\\
=&\dfrac{1}{2\pi}\left(\dfrac{\log^2(n)+2\log(n)+2}{n}-\dfrac{\log^2(m)+2\log(m)+2}{m}\right)
\end{aligned}
$$
\par
根据$$
\lim_{n \to \infty}\dfrac{\log^2(n)+2\log(n)+2}{n}=0
$$
\par
对任意$\varepsilon>0$,存在$N\in \mathbb{N}$,使得任意$m,n>N$,有
$$
\dfrac{1}{2\pi}\int_{0}^{2\pi}|f_m(\theta)-f_n(\theta)|^2\mathrm{d}\theta<\varepsilon
$$
\par
则$\{f_n\}$是$\mathcal{R}$中的Cauchy列,但是$f\to +\infty, \theta \to 0^{+}$,
则$f \notin \mathcal{R} $.
\end{solution}



\begin{problem}
   Exercise 7
   \par 
    证明: 三角级数
    $$
     \sum_{n\geqslant 2}\dfrac{1}{\log n}\sin nx
    $$
    \par
    对任意的$x$都收敛,但不是任何Riemann可积函数的Fourier级数.
\end{problem}

\begin{solution}
\par
$\dfrac{1}{\log n}$单调递减趋于$0$,并且$\sum_{n=2}^{N}\sin nx$一致有界.
根据Dirichlet判别法,$x\ne 0$时,此级数收敛.
\par
$x=0$时,此级数显然也收敛.
\par
改写此级数为:
$$
\sum_{n\geqslant 2}\dfrac{1}{2i\log n}\mathrm{e}^{inx}
+\sum_{n\geqslant 2}\dfrac{-1}{2i\log n}\mathrm{e}^{-inx}
$$
\par
我们用反证法,假设其为Riemann可积函数$f(x)$的Fourier级数,那么根据
Parseval恒等式:
\par
$$
\dfrac{1}{2\pi}\int_{0}^{2\pi}|f(x)|^2\mathrm{d}x=\sum_{n=2}^{+\infty}\dfrac{1}{2\log^2n}
$$
\par
等号左边应该为有限数,而右边发散到正无穷,矛盾!不存在满足条件的Riemann可积函数.
\end{solution}



\begin{problem}
    Exercise 8
    \par
    上一章我们讨论了$\displaystyle{\sum_{n=1}^{\infty}\dfrac{1}{n^2}}$的问题.接下来
    我们进一步讨论.
    \par
    (1)设$f(\theta)=|\theta|, \theta\in [\pi,\pi]$.
    利用Parseval恒等式计算:
    $$
    \sum_{n=0}^{\infty}\dfrac{1}{(2n+1)^4},\quad \sum_{n=1}^{\infty}\dfrac{1}{n^4}
    $$
    \par
    (2)考虑以$2\pi$为周期的奇函数$f(\theta)=\theta(\pi-\theta)$.证明:
    $$
    \sum_{n=0}^{\infty}\dfrac{1}{(2n+1)^6}=\dfrac{\pi^6}{960},\quad
    \sum_{n=1}^{\infty}\dfrac{1}{n^6}=\dfrac{\pi^6}{945}
    $$
\end{problem}

\begin{solution}
\par
(1)计算得到:
$$
f(\theta)\sim \dfrac{\pi}{2} +\sum_{n \text{ odd}}\dfrac{-2}{\pi n^2}\mathrm{e}^{inx}
$$
\par
利用Parseval恒等式:
$$
\dfrac{\pi^2}{4}+2\cdot\sum_{n=0}^{\infty}\dfrac{4}{\pi^2(2n+1)^4}
=\dfrac{1}{2\pi}\int_{-\pi}^{\pi}\theta^2\mathrm{d}\theta
$$
\par
整理得到:$$
\sum_{n=0}^{\infty}\dfrac{1}{(2n+1)^4}=\dfrac{\pi^4}{96}
$$
\par
$$
\sum_{n=1}^{\infty}\dfrac{1}{n^4}
=\sum_{n=0}^{\infty}\dfrac{1}{(2n+1)^4}+\sum_{n=1}^{\infty}\dfrac{1}{(2n)^4}
$$
\par
整理得到:
$$
\sum_{n=1}^{\infty}\dfrac{1}{n^4}
=\dfrac{16}{15}\sum_{n=0}^{\infty}\dfrac{1}{(2n+1)^4}
=\dfrac{\pi^4}{90}
$$
\par
(2)计算得到:
$$
f(\theta)\sim \sum_{n \text{ odd}}\dfrac{4}{i\pi n^3}\mathrm{e}^{inx}
$$
\par
利用Parseval恒等式:
$$
2\cdot\sum_{n=0}^{\infty}\dfrac{16}{\pi^2(2n+1)^6}
=\dfrac{1}{\pi}\int_{0}^{\pi}\theta^2(\pi-\theta)^2\mathrm{d}\theta
$$
\par
整理得到:$$
\sum_{n=0}^{\infty}\dfrac{1}{(2n+1)^6}=\dfrac{\pi^6}{960}
$$
\par
$$
\sum_{n=1}^{\infty}\dfrac{1}{n^6}
=\sum_{n=0}^{\infty}\dfrac{1}{(2n+1)^6}+\sum_{n=1}^{\infty}\dfrac{1}{(2n)^6}
$$
\par
整理得到:
$$
\sum_{n=1}^{\infty}\dfrac{1}{n^6}
=\dfrac{64}{63}\sum_{n=0}^{\infty}\dfrac{1}{(2n+1)^6}
=\dfrac{\pi^6}{945}
$$
\end{solution}





\begin{problem}
   Exercise 9
   \par
   证明:当$\alpha \in \mathbb{R}$不是整数时,
   $\dfrac{\pi}{\sin \pi \alpha}\mathrm{e}^{i(\pi-x)\alpha},x\in [0,2\pi]$ 
   的Fourier级数为:
   $$
    \sum_{n=-\infty}^{+\infty}\dfrac{\mathrm{e}^{inx}}{n+\alpha}
   $$
   \par
   利用Parseval恒等式证明:
   $$
   \sum_{n=-\infty}^{+\infty}\dfrac{1}{(n+\alpha)^2}=\dfrac{\pi^2}{(\sin \pi \alpha)^2}
   $$
\end{problem}

\begin{solution}
\par
证明:
\par
计算得到:
\par
$$
\begin{aligned}
\hat{f}(n)&=\dfrac{1}{2\pi}\int_{0}^{2\pi}
\dfrac{\pi}{\sin \pi \alpha}\mathrm{e}^{i(\pi-x)\alpha}\mathrm{e}^{-inx} \mathrm{d}x\\
&=\dfrac{1}{2\sin \pi\alpha}\mathrm{e}^{i\pi \alpha}
\int_{0}^{2\pi}\mathrm{e}^{-i(n+\alpha)x}\mathrm{d}x\\
&=\left. \dfrac{1}{2\sin \pi\alpha}\mathrm{e}^{i\pi \alpha}
\dfrac{\mathrm{e}^{-i(n+\alpha)x}}{-i(n+\alpha)}\right|^{2\pi}_{0}\\
&=\dfrac{1}{n+\alpha}
\end{aligned}
$$
\par
于是:
$$
\dfrac{\pi}{\sin \pi \alpha}\mathrm{e}^{i(\pi-x)\alpha}
\sim \sum_{n=-\infty}^{+\infty}\dfrac{\mathrm{e}^{inx}}{n+\alpha}
$$
\par
根据Parseval恒等式:
$$
\sum_{n=-\infty}^{+\infty}\dfrac{1}{(n+\alpha)^2}
=\dfrac{1}{2\pi}\int_{0}^{2\pi}\dfrac{\pi^2}{\sin^2 \pi\alpha }\mathrm{d}x
=\dfrac{\pi^2}{\sin^2\pi\alpha}
$$
\par
即
$$
\sum_{n=-\infty}^{+\infty}\dfrac{1}{(n+\alpha)^2}
=\dfrac{\pi^2}{(\sin\pi\alpha)^2}
$$


\end{solution}


\begin{problem}
    Exercise 10
    \par
    考虑上一章讨论过的弦振动问题,位移$u(x,t)$满足:
    $$
   \dfrac{1}{c^2}\dfrac{\partial^2u}{\partial t^2}
   =\dfrac{\partial^2 u}{\partial x^2},
   \quad c^2=\dfrac{\tau}{\rho}
    $$
    \par
    弦振动初始条件为:
    $$
    u(x,0)=f(x), \quad \dfrac{\partial u}{\partial t}(x,0)=g(x)
    $$
    \par
    且$f\in C^1,g\in C$.
    \par
    我们定义此弦的总能量为:
    \par
    $$
    E(t)=\dfrac{1}{2}\rho \int_{0}^{L}\left(\dfrac{\partial u}{\partial t}\right)^2\mathrm{d}x
    +\dfrac{1}{2}\tau \int_{0}^{L}\left(\dfrac{\partial u}{\partial x}\right)^2\mathrm{d}x
    $$
    \par
    证明: 总能量守恒,并且:
    $$
    E(t)=E(0)=\dfrac{1}{2}\rho \int_{0}^{L}g^2(x)\mathrm{d}x
    +\dfrac{1}{2}\tau \int_{0}^{L}f'^2(x)\mathrm{d}x
    $$
\end{problem}

\begin{solution}
证明:
\par
 对总能量求导,那么根据题设,对正则性良好诸函数可以交换积分和求导次序:
 $$
 \begin{aligned}
 &\dfrac{\mathrm{d} E}{\mathrm{d} t}\\
 =&\dfrac{1}{2}\int_{0}^{L}\rho \cdot 2
 \dfrac{\partial u}{\partial t}\cdot \dfrac{\partial^2 u}{\partial t^2}\mathrm{d}x
 +\dfrac{1}{2}\int_{0}^{L}\tau \cdot 2
 \dfrac{\partial u}{\partial x}\cdot \dfrac{\partial^2 u}{\partial x\partial t}
 \mathrm{d}x\\
 =&\int_{0}^{L}\rho 
 \dfrac{\partial u}{\partial t}\cdot \dfrac{\partial^2 u}{\partial t^2}\mathrm{d}x
 +\int_{0}^{L}\tau
 \dfrac{\partial u}{\partial x}\cdot \dfrac{\partial^2 u}{\partial x\partial t}
 \mathrm{d}x\\
 =&\int_{0}^{L}\rho 
 \dfrac{\partial u}{\partial t}\cdot \dfrac{\partial^2 u}{\partial t^2}\mathrm{d}x
 +\left. \tau \dfrac{\partial u}{\partial x}\dfrac{\partial u}{\partial t}\right|^L_0
 -\int_{0}^{L}\tau \dfrac{\partial u}{t}\dfrac{\partial ^2 u}{\partial x^2}\mathrm{d}x\\
 =&\int_{0}^{L}\dfrac{\partial u}{\partial t}\left(\rho\dfrac{\partial^2 u}{\partial t^2}
 -\tau \dfrac{\partial^2 u}{\partial x^2}\right)\mathrm{d}x\\
 =&0
 \end{aligned}
 $$
 \par
 其中用到$\dfrac{\partial u}{\partial t}(0,t)=\dfrac{\partial u}{\partial t}(L,t)$,
 以及$$
 \rho\dfrac{\partial^2 u}{\partial t^2}
 =\tau \dfrac{\partial^2 u}{\partial x^2}
 $$
\par
则
 $$
    E(t)=E(0)=\dfrac{1}{2}\rho \int_{0}^{L}g^2(x)\mathrm{d}x
    +\dfrac{1}{2}\tau \int_{0}^{L}f'^2(x)\mathrm{d}x
    $$

\end{solution}



\begin{problem}
    Exercise 11
    \par
    Wirtinger不等式和$\text{Poincar}\acute{\text{e}}$不等式建立了函数范数和其导数范数的
    关系.
    \par
    (1)如果$f$周期为$T$,连续且满足分段$C^1$且$\int_{0}^{T}f(t)\mathrm{d}t=0$,证明:
     $$
     \int_{0}^{T}|f(t)|^2\mathrm{d}t\leqslant \dfrac{T^2}{4\pi^2}\int_{0}^{T}|f'(t)|^2\mathrm{d}t
     $$
     \par
     等号当且仅当$f(t)=A\sin \left(\dfrac{2\pi t}{T}\right)
     +B\cos\left(\dfrac{2\pi t}{T}\right)$时取得.
     \par
     (2)如果$f$同上,$g$是$C^1$且周期为$T$的函数,证明:
     $$
     \left|\int_{0}^{T}\overline{f(t)}g(t)\mathrm{d}t\right|^2\leqslant 
     \dfrac{T^2}{4\pi^2}\int_{0}^{T}|f(t)|^2\mathrm{d}t\int_{0}^{T}
     |g'(t)|^2 \mathrm{d}t
     $$
     \par
     (3)对于任意紧致区间$[a,b]$和任意满足$f(a)=f(b)=0$的连续可导函数$f$,证明:
     $$
     \int_{a}^{b}|f(t)|^2\mathrm{d}t\leqslant \dfrac{(b-a)^2}{\pi^2}\int_{a}^{b}
     |f'(t)|^2\mathrm{d}t
     $$
     \par
     讨论等号成立,系数$\dfrac{(b-a)^2}{\pi^2}$不可改进
\end{problem}

\begin{solution}
\par
(1)
\par
我们考虑一般系数的Fourier级数,
$$
\int_{0}^{T}f(t)\mathrm{d}t=0
$$
\par
则$\hat{f}(0)=0$, 那么
$$
f(t)\sim \sum_{n \ne 0}a_n\mathrm{e}^{in\frac{2\pi}{T}t}
$$
\par
其中$a_n=\hat{f}(n)$.
\par
根据$f$周期为$T$,那么$f(0)=f(T)$,且由于分段$C_1$且连续,除了有限个点外$f$可微,
且$f'$可积且绝对可积,
那么根据Fourier逐项求导定理:
$$
f'(t)\sim \sum_{n \ne 0}in\dfrac{2\pi}{T}a_n\mathrm{e}^{in\frac{2\pi}{T}t} 
$$
\par
根据Parseval恒等式:
$$
\begin{aligned}
&\int_{0}^{T}|f(t)|^2\mathrm{d}t=2\pi\sum_{n \ne 0}|a_n|^2\\
&\int_{0}^{T}|f'(t)|^2\mathrm{d}t=\dfrac{8\pi^2}{T^2}\sum_{n \ne 0}n^2|a_n|^2
\end{aligned}
$$
\par
原命题等价于
$$
\sum_{|n|\geqslant 2}|a_n|^2\leqslant \sum_{|n| \geqslant 2}n^2|a_n|^2
$$
\par
显然成立,若取得等号,则有:
$$
a_n=0, |n|\geqslant 2
$$
\par
有$$
\begin{aligned}
f(t)
=&a_1\mathrm{e}^{i\frac{2\pi}{T}t}+a_{-1}\mathrm{e}^{-i\frac{2\pi}{T}t}\\
=&A\sin \left(\dfrac{2\pi}{T}t\right)+B\cos \left(\dfrac{2\pi}{T}t\right)
\end{aligned}
$$
\par
其中
$A=i(a_1-a_{-1}),B=a_1+a_{-1}$
\par
(2)根据条件可以知道$$
\int_{0}^{T}\overline{f(t)}\mathrm{d}t=0
$$
\par
那么我们只需要研究:
$$
\left|\int_{0}^{T}\overline{f(t)}\left(g(t)-\dfrac{1}{T}\int_{0}^{T}g(x)\mathrm{d}x\right)\right|^2
$$
\par
由于
$$
\int_{0}^{T}\left(g(t)-\dfrac{1}{T}\int_{0}^{T}g(x)\mathrm{d}x\right)\mathrm{d}t=0
$$
\par
先用Cauchy Schwarz不等式,再用$(1)$的Wirtinger不等式:
$$
\begin{aligned}
&\left|\int_{0}^{T}\overline{f(t)}g(t)\mathrm{d}t\right|^2\\
=&\left|\int_{0}^{T}\overline{f(t)}\left(g(t)-\dfrac{1}{T}\int_{0}^{T}g(x)\mathrm{d}x\right)\right|^2\\
\leqslant& \int_{0}^{T}|f(t)|^2\mathrm{d}t\int_{0}^{T}\left|g(t)-\dfrac{1}{T}
\int_{0}^{T}g(x)\mathrm{d}x\right|^2\mathrm{d}t\\
\leqslant &\int_{0}^{T}|f(t)|^2\mathrm{d}t \cdot \dfrac{T^2}{4\pi^2}\int_{0}^{T}|g'(t)|^2\mathrm{d}t\\
= & \dfrac{T^2}{4\pi^2}\int_{0}^{T}|f(t)|^2\mathrm{d}t \cdot \int_{0}^{T}|g'(t)|^2\mathrm{d}t
\end{aligned}
$$
\par
可见成立.
\par
(3)
我们做变换$t=\dfrac{T}{2}\cdot\dfrac{x-a}{b-a}$,其将$[a,b]$映射为$[0,T]$,
我们继续将$f(x)$关于$x=a$奇延拓到$[2a-b,b]$,使得$f(x)=-f(2a-x),x\in [2a-b,a]$.
\par
则可设$g(t)=f(x)$,$g(t)$为满足连续可导的奇函数,
$$
\int_{-\frac{T}{2}}^{\frac{T}{2}}g(t)\mathrm{d}t=0
$$
\par
利用$(1)$的Wirtinger不等式有:
$$
\begin{aligned}
&\int_{-\frac{T}{2}}^{\frac{T}{2}}|g(t)|^2\mathrm{d}t
\leqslant \dfrac{T^2}{4\pi^2}\int_{-\frac{T}{2}}^{\frac{T}{2}}|g'(t)|^2\mathrm{d}t\\
&\int_{0}^{\frac{T}{2}}|g(t)|^2\mathrm{d}t
\leqslant \dfrac{T^2}{4\pi^2}\int_{0}^{\frac{T}{2}}|g'(t)|^2\mathrm{d}t
\end{aligned}
$$
\par
当且仅当$g(t)=A\sin\left(\dfrac{2\pi t}{T}\right)$取得等号,
由于奇函数性质,此时余弦项系数为$0$.
\par
接下来换回$f(x)$有:
$$
\begin{aligned}
\int_{a}^{b}|f(x)|^2\dfrac{T}{2(b-a)}\mathrm{d}x
&\leqslant \dfrac{T^2}{4\pi^2}\int_{a}^{b}\left|f'(x)\cdot \dfrac{\mathrm{d}x}{\mathrm{d}t}\right|^2
\dfrac{T}{2(b-a)}\mathrm{d}x\\
\int_{a}^{b}|f(x)|^2\mathrm{d}x
&\leqslant \dfrac{T^2}{4\pi^2}\int_{a}^{b}\left|f'(x)\right|^2\dfrac{4(b-a)^2}{T^2}\mathrm{d}x\\
\int_{a}^{b}|f(x)|^2\mathrm{d}x
&\leqslant \dfrac{(b-a)^2}{\pi^2}\int_{a}^{b}\left|f'(x)\right|^2\mathrm{d}x\\
\end{aligned}
$$
\par
等号当且仅当$f(x)=A\sin\left(\dfrac{\pi(x-a)}{b-a}\right)$
,可见系数不能再改进.
\end{solution}



\begin{problem}
Exercise 12:
\par
证明:
$$
\int_{0}^{\infty}\dfrac{\sin x}{x}\mathrm{d}x=\dfrac{\pi}{2}
$$
\end{problem}

\begin{solution}
\par
证明:
\par
我们考虑Dirichlet核,
$$
\dfrac{1}{2\pi}\int_{-\pi}^{\pi}D_N(x)\mathrm{d}x=1
$$
\par
则
$$
\int_{-\pi}^{\pi}\dfrac{\sin (N+\frac{1}{2})x}{\sin \frac{1}{2}x}\mathrm{d}x=2\pi
$$
\par
那么拆开有:
$$
\begin{aligned}
\int_{-\pi}^{\pi}\dfrac{\sin (N+\frac{1}{2})x}{\frac{1}{2}x}\mathrm{d}x
+\int_{-\pi}^{\pi}\sin Nx \cdot \cos \frac{1}{2}x\cdot T(x)\mathrm{d}x
+\int_{-\pi}^{\pi}\cos Nx \cdot \sin \frac{1}{2}x\cdot T(x)\mathrm{d}x
=2\pi\\
2\int_{-(N+\frac{1}{2})\pi}^{(N+\frac{1}{2})\pi}\dfrac{\sin t}{t}\mathrm{d}t
+\int_{-\pi}^{\pi}\sin Nx \cdot \cos \frac{1}{2}x\cdot T(x)\mathrm{d}x
+\int_{-\pi}^{\pi}\cos Nx \cdot \sin \frac{1}{2}x\cdot T(x)\mathrm{d}x
=2\pi\\
4\int_{0}^{(N+\frac{1}{2})\pi}\dfrac{\sin t}{t}\mathrm{d}t
+\int_{-\pi}^{\pi}\sin Nx \cdot \cos \frac{1}{2}x\cdot T(x)\mathrm{d}x
+\int_{-\pi}^{\pi}\cos Nx \cdot \sin \frac{1}{2}x\cdot T(x)\mathrm{d}x
=2\pi\\
\end{aligned}
$$
\par
其中$T(x)=\dfrac{1}{\sin \frac{1}{2}x}-\dfrac{1}{\frac{1}{2}x}$,而且
$$
\lim_{x \to 0}T(x)=0
$$
\par
可延拓使之连续,于是满足Riemann Lebesgue引理:
$$
\begin{aligned}
\lim_{N \to \infty}
\int_{-\pi}^{\pi}\sin Nx \cdot \cos \frac{1}{2}x\cdot T(x)\mathrm{d}x=0\\
\lim_{N \to \infty}
\int_{-\pi}^{\pi}\cos Nx \cdot \sin \frac{1}{2}x\cdot T(x)\mathrm{d}x=0\\
\end{aligned}
$$
\par
则对之前等式, 令$N\to \infty$,有:
$$
4\int_{0}^{\infty}\dfrac{\sin x}{x}\mathrm{d}x=2\pi
$$
\par
则$$
\int_{0}^{\infty}\dfrac{\sin x}{x}\mathrm{d}x=\dfrac{\pi}{2}
$$
\end{solution}



\begin{problem}
    Exercise 13
    \par
    设$f$是周期函数而且是$C^k$的,证明:
    $$
    \hat{f}(n)=o\left(\dfrac{1}{|n|^k}\right)
    $$
\end{problem}

\begin{solution}
\par
证明:
\par
首先根据$f$是周期函数且是$C^k$的,设周期为$T$可以证明:
\par
$$
\hat{f}^{(k)}(n)=\left(in\frac{2\pi}{T}\right)^{k}\hat{f}(n)
$$
\par
则
$$
\begin{aligned}
||n|^k\cdot \hat{f}(n)|
=&\dfrac{|n|^k}{\left|\left(in\dfrac{2\pi}{T}\right)^k\right|}|\hat{f}^{(k)}(n)|\\
=&C_{\pi,T,k}|\hat{f}^{(k)}(n)|\\
=&C_{\pi,T,k}\cdot \left|\dfrac{1}{2\pi}
\int_{0}^{T}f^{(k)}(x)\mathrm{e}^{-in\frac{2\pi}{T}{x}}\mathrm{d}x\right|\\
=&\widetilde{C}_{\pi,T,k}\cdot \left|\dfrac{1}{2\pi}
\int_{0}^{2\pi}f^{(k)}(t)\mathrm{e}^{-int}\mathrm{d}t\right|\\
\leqslant & \widetilde{C}_{\pi,T,k}\cdot \left|\dfrac{1}{2\pi}
\int_{0}^{2\pi}f^{(k)}(t)\cos(nt)\mathrm{d}t\right|\\
+&\widetilde{C}_{\pi,T,k}\cdot \left|\dfrac{1}{2\pi}
\int_{0}^{2\pi}f^{(k)}(t)\sin(nt)\mathrm{d}t\right|\\
\end{aligned}
$$
\par
$f^{(k)}$连续,那么应用Riemann Lebesgue引理,
$$
\begin{aligned}
&\lim_{n \to \infty} \int_{0}^{2\pi}f^{(k)}(t)\sin(nt)\mathrm{d}t=0\\
&\lim_{n \to \infty}\int_{0}^{2\pi}f^{(k)}(t)\cos(nt)\mathrm{d}t=0
\end{aligned}
$$
\par
于是
$$
\lim_{n \to \infty}|n|^k\hat{f}(n)=0
$$
\par
就是
$$
\hat{f}(n)=o\left(\dfrac{1}{|n|^k}\right)
$$
\end{solution}



\begin{problem}
    Exercise 14
    \par
    证明: 圆周上$C^1$函数$f$的Fourier级数绝对收敛.
\end{problem}

\begin{solution}
\par
证明:
即证明$\sum_{n\in \mathbb{Z}}\left|\hat{f}(n)\right|$收敛.
\par
既然$f\in C^1$,那么:
$$
\hat{f}'(n)=in\hat{f}(n)
$$
\par
进而
$$
\left|\hat{f}(n)\right|=\left|\dfrac{\hat{f}'(n)}{n}\right|
$$
\par
利用Cauchy Schwarz不等式:
$$
\begin{aligned}
&\left|\hat{f}(0)\right|+\sum_{n\ne 0}\left|\hat{f}(n)\right|\\
=&\left|\hat{f}(0)\right|+\sum_{n\ne 0}\left|\hat{f}'(n)\cdot \overline{\dfrac{1}{n}}\right|\\
\leqslant & 
\left|\hat{f}(0)\right|+\left(\sum_{n\ne 0}\left|\hat{f}'(n)\right|^2\right)^{\frac{1}{2}}
\cdot \left(\sum_{n \ne 0}\dfrac{1}{n^2}\right)^{\frac{1}{2}}\\
\leqslant & \left|\hat{f}(0)\right|+\left(\sum_{n \in \mathbb{Z}}\left|\hat{f}'(n)\right|^2\right)^{\frac{1}{2}}
\cdot \left(\sum_{n \ne 0}\dfrac{1}{n^2}\right)^{\frac{1}{2}}\\
=&\left|\hat{f}(0)\right|+\left(\dfrac{1}{2\pi}\int_{0}^{2\pi}\left|{f}'(x)\right|^2\mathrm{d}x\right)^{\frac{1}{2}}
\cdot \left(\sum_{n \ne 0}\dfrac{1}{n^2}\right)^{\frac{1}{2}}\\
=&\left|\hat{f}(0)\right|+\dfrac{\pi}{\sqrt{3}}\left(\dfrac{1}{2\pi}\int_{0}^{2\pi}\left|{f}'(x)\right|^2\mathrm{d}x\right)^{\frac{1}{2}}
\\
<&\infty
\end{aligned}
$$
\par
其中用到Parseval恒等式,并且由于$f\in C^1 $,推理自明.
\par
原命题成立.
\end{solution}



\begin{problem}
        Exercise 15
        \par
        设$f$周期为$2\pi$且在$[-\pi,\pi]$上Riemann可积.
        \par
        (1)证明:$$
        \hat{f}(n)=-\dfrac{1}{2\pi}\int_{-\pi}^{\pi}
        f\left(x+\dfrac{\pi}{n}\right)\mathrm{e}^{-inx}\mathrm{d}x
        $$
        \par
        则
        $$
        \hat{f}(n)=\dfrac{1}{4\pi}\int_{-\pi}^{\pi}\left[f(x)-
        f\left(x+\dfrac{\pi}{n}\right)\right]\mathrm{e}^{-inx}\mathrm{d}x
        $$
        \par
        (2)假设$f$满足$\alpha$阶的H$\ddot{\text{o}}$lder条件,
        即对于常数$0<\alpha \leqslant 1$和$C>0$,有:
        $$
        |f(x+h)-f(x)|\leqslant C|h|^{\alpha}, \forall x,h 
        $$
        \par
        利用(1)证明:
        $$
        \hat{f}(n)=O\left(\dfrac{1}{|n|^{\alpha}}\right)
        $$
        \par
        (3)
        证明:(2)中结果无法改进,通过说明:
        $$
         f(x)=\sum_{k=0}^{\infty}2^{-k\alpha}\mathrm{e}^{i2^kx}
        $$
        \par
        满足:
        $$
        |f(x+h)-f(x)|\leqslant C|h|^{\alpha}
        $$
        \par
        其中$0<\alpha<1$,
        且
        $$
         \hat{f}(N)=\dfrac{1}{N^{\alpha}}
        $$
        \par
        当$N=2^k$成立.
\end{problem}

\begin{solution}
\par
(1)证明:
$$
\hat{f}(n)=\dfrac{1}{2\pi}\int_{-\pi}^{\pi}f(x)\mathrm{e}^{-inx}\mathrm{d}x
$$
\par
换元$t+\dfrac{\pi}{n}=x$,则:
$$
\begin{aligned}
\hat{f}(n)&=\dfrac{1}{2\pi}\int_{-\pi}^{\pi}
f\left(t+\dfrac{\pi}{n}\right)\mathrm{e}^{-int-i\pi}\mathrm{d}t\\
&=-\dfrac{1}{2\pi}\int_{-\pi}^{\pi}
f\left(t+\dfrac{\pi}{n}\right)\mathrm{e}^{-int}\mathrm{d}t\\
&=-\dfrac{1}{2\pi}\int_{-\pi}^{\pi}
f\left(x+\dfrac{\pi}{n}\right)\mathrm{e}^{-inx}\mathrm{d}x\\
\end{aligned}
$$
\par
接着各取一半:
$$
\hat{f}(n)=\dfrac{1}{4\pi}\int_{-\pi}^{\pi}\left[f(x)-
f\left(x+\dfrac{\pi}{n}\right)\right]\mathrm{e}^{-inx}\mathrm{d}x
$$
\par
(2)证明:
\par
利用(1)及H$\ddot{\text{o}}$lder条件:
$$
\begin{aligned}
\left|\hat{f}(n)\right|
=&\dfrac{1}{4\pi}\left|\int_{-\pi}^{\pi}\left[f(x)-
f\left(x+\dfrac{\pi}{n}\right)\right]\mathrm{e}^{-inx}\mathrm{d}x\right|\\
\leqslant & \dfrac{1}{4\pi}\int_{-\pi}^{\pi}\left|f(x)-
f\left(x+\dfrac{\pi}{n}\right)\right|\mathrm{d}x\\
\leqslant & \dfrac{1}{4\pi}\int_{-\pi}^{\pi}C\left|\dfrac{\pi}{n}\right|^{\alpha}\mathrm{d}x\\
= & \dfrac{C}{2}\cdot \dfrac{\pi^\alpha}{|n|^{\alpha}}
\end{aligned}
$$
\par
于是我们得到:
$$
\hat{f}(n)=O\left(\dfrac{1}{|n|^{\alpha}}\right)
$$
\par
(3)
证明:
考虑直接作差
\par
$$
\begin{aligned}
f(x)&=\sum_{k=0}^{\infty}2^{-k\alpha}\mathrm{e}^{i2^kx}\\
f(x+h)&=\sum_{k=0}^{\infty}2^{-k\alpha}\mathrm{e}^{i2^k(x+h)}
\end{aligned}
$$
\par
有:
$$
\begin{aligned}
|f(x+h)-f(x)|
=&\left|\sum_{k=0}^{\infty}2^{-k\alpha}
\left(\mathrm{e}^{i2^k(x+h)}-\mathrm{e}^{i2^kx}\right)\right|\\
\leqslant& \sum_{2^k|h|\leqslant 1} 2^{-k\alpha}\left|
\mathrm{e}^{i2^k(x+h)}-\mathrm{e}^{i2^kx}\right|\\
+&\sum_{2^k|h|> 1} 2^{-k\alpha}\left|
\mathrm{e}^{i2^k(x+h)}-\mathrm{e}^{i2^kx}\right|\\
\leqslant& \sum_{2^k|h|\leqslant 1} 2^{-k\alpha}\left|
\mathrm{e}^{i2^kh}-1\right|
+\sum_{2^k|h|> 1} 2\cdot 2^{-k\alpha}\\
\leqslant& \sum_{2^k|h|\leqslant 1} 2^{-k\alpha}\cdot 2^k|h|
+\sum_{2^k|h|> 1} 2\cdot 2^{-k\alpha}\\
\leqslant& \dfrac{1}{1-2^{\alpha-1}}h|^{\alpha}
+ \dfrac{2}{1-2^{-\alpha}}|h|^{\alpha}\\
\leqslant & C_{\alpha}|h|^{\alpha}
\end{aligned}
$$
\par
即
$$
|f(x+h)-f(x)|\leqslant C_{\alpha}|h|^{\alpha}
$$
\par
有:
$$
\sum_{n=0}^{\infty}\left|2^{-k\alpha}\mathrm{e}^{i2^kx}\right|=\sum_{n=0}^{\infty}2^{-k\alpha}<+\infty
$$
\par
根据Weierstrass判别法,$$
f(x)=\sum_{n=0}^{\infty}2^{-k\alpha}\mathrm{e}^{i2^kx}
$$
\par
一致收敛,那么交换积分和求和次序,可以得到Fourier系数为:
$$
\hat{f}(N)=\dfrac{1}{N^{\alpha}}, \quad N=2^k, k \in \mathbb{Z}_{\geqslant 0}
$$
\par
这就说明了原估计不可以改进.
\end{solution}



\begin{problem}
    Exercise 16
    \par
    设$f$周期为$2\pi$,对常数$K>0$满足Lipschitz条件:
    $$
    |f(x)-f(y)|\leqslant K|x-y|, \forall x,y
    $$
    \par
    下证, $f$的Fourier级数绝对收敛且一致收敛.
    \par
    (1)对任意$h>0$,我们定义:
    $$
    g_h(x)=f(x+h)-f(x-h)
    $$
    \par
    证明:
    $$
    \dfrac{1}{2\pi}\int_{0}^{2\pi}|g_h(x)|^2\mathrm{d}x=\sum_{n=-\infty}^{\infty}
    4|\sin nh|^2\left|\hat{f}(n)\right|^2
    $$
    \par
    以及:
    $$
     \sum_{n-\infty}^{\infty}|\sin nh|^2\left|\hat{f}(n)\right|^2
     \leqslant K^2h^2
    $$
    \par
    (2)设$p$是一个正整数, 令$h=\dfrac{\pi}{2^{p+1}}$,证明:
    $$
    \sum_{2^{p-1}<|n|\leqslant 2^p}\left|\hat{f}(n)\right|^2
    \leqslant \dfrac{K^2\pi^2}{2^{2p+1}}
    $$
    \par
    (3)估计$$
    \sum_{2^{p-1}<|n|\leqslant 2^p}\left|\hat{f}(n)\right|
    $$
    \par
    并且证明: $f$的Fourier级数绝对收敛,进而一致收敛
    \par
    (4)修改上述过程,可以证明Bernstein定理,即若$f$满足H$\ddot{\text{o}}$lder条件,且
    阶数$\alpha>\dfrac{1}{2}$,那么其Fourier级数绝对收敛.
\end{problem}

\begin{solution}
\par
(1)证明:
\par
$$
\begin{aligned}
\hat{g}_h(n)=&\dfrac{1}{2\pi}\int_{0}^{2\pi}f(x+h)\mathrm{e}^{-inx}\mathrm{d}x
-\dfrac{1}{2\pi}\int_{0}^{2\pi}f(x-h)\mathrm{e}^{-inx}\mathrm{d}x\\
=& \dfrac{1}{2\pi}\int_{0}^{2\pi}f(t)\mathrm{e}^{-int}\mathrm{e}^{inh}-
\dfrac{1}{2\pi}\int_{0}^{2\pi}f(t)\mathrm{e}^{-int}\mathrm{e}^{-inh}\mathrm{d}t\\
=&2i\sin nh \cdot \hat{f}(n)
\end{aligned}
$$
\par
$f(x)$满足Lipschitz条件,则连续,则Riemann可积,
根据Parseval恒等式:
$$
\dfrac{1}{2\pi}\int_{0}^{2\pi}\left|g_h(x)\right|^2\mathrm{d}x=\sum_{n=-\infty}^{+\infty}
4\left|\sin nh\right|^2\left|\hat{f}(n)\right|^2
$$
\par
进而,
$$
\begin{aligned}
\sum_{n=-\infty}^{+\infty}
4\left|\sin nh\right|^2\left|\hat{f}(n)\right|^2
\leqslant &\dfrac{1}{2\pi}\int_{0}^{2\pi}\left|g_h(x)\right|^2\mathrm{d}x\\
\leqslant &\dfrac{1}{2\pi}\int_{0}^{2\pi}4K^2h^2\mathrm{d}x\\
= &4K^2h^2
\end{aligned}
$$
\par
于是我们得到:
$$
\sum_{n=-\infty}^{+\infty}
\left|\sin nh\right|^2\left|\hat{f}(n)\right|^2
\leqslant  K^2h^2
$$
\par
(2)我们根据$(1)$有:
$$
\begin{aligned}
\sum_{2^{p-1}<|n|\leqslant 2^p}|\sin nh|^2\left|\hat{f}(n)\right|^2
\leqslant
\sum_{n=-\infty}^{+\infty}
\left|\sin nh\right|^2\left|\hat{f}(n)\right|^2
\leqslant  K^2h^2
\end{aligned}
$$
\par
取$h=\dfrac{\pi}{2^{p+1}}$,则
$$
\begin{aligned}
    \sum_{2^{p-1}<|n|\leqslant 2^p}|\sin nh|^2\left|\hat{f}(n)\right|^2
    \leqslant  K^2\dfrac{\pi^2}{2^{2p+2}}
    \end{aligned}
$$
\par
而此时$nh \in \left(\dfrac{\pi}{4},\dfrac{\pi}{2}\right]$
\par
则
$$
\begin{aligned}
    \sum_{2^{p-1}<|n|\leqslant 2^p}\dfrac{1}{2}\cdot\left|\hat{f}(n)\right|^2
    \leqslant  K^2\dfrac{\pi^2}{2^{2p+2}}
    \end{aligned}
$$
\par
整理得到:
$$
\sum_{2^{p-1}<|n|\leqslant 2^p}\left|\hat{f}(n)\right|^2
    \leqslant  K^2\dfrac{\pi^2}{2^{2p+1}}
$$
\par
(3)我们直接估计,
$$
\begin{aligned}
\sum_{2^{p-1}<|n|\leqslant 2^p}\left|\hat{f}(n)\right|
    \leqslant & \left(\sum_{2^{p-1}<|n|\leqslant 
    2^p}\left|\hat{f}(n)\right|^2\right)^{\frac{1}{2}}
    \cdot \left(\sum_{2^{p-1}<|n|\leqslant 2^p}1^2\right)^{\frac{1}{2}}\\
    \leqslant & K\dfrac{\pi}{2^{p+\frac{1}{2}}}\cdot 2^{\frac{p}{2}-\frac{1}{2}}\\
    = & \dfrac{K\pi}{2}{2^{-\frac{p}{2}}}\\
\end{aligned}
$$
\par
进一步,
$$
\begin{aligned}
\sum_{n=-\infty}^{+\infty}\left|\hat{f}(n)\right|
=&\sum_{|n|\leqslant 1}\left|\hat{f}(n)\right|
+\sum_{p\geqslant 1}\sum_{2^{p-1}<|n|\leqslant 2^p}\left|\hat{f}(n)\right|\\
\leqslant & \sum_{|n|\leqslant 1}\left|\hat{f}(n)\right|
+ \sum_{p\geqslant 1} \dfrac{K\pi}{2}2^{-\frac{p}{2}}\\
=& \sum_{|n|\leqslant 1}\left|\hat{f}(n)\right|
+ \dfrac{K\pi}{2}\cdot \dfrac{1}{1-2^{-\frac{1}{2}}}\\
<&+\infty
\end{aligned}
$$
\par
则其Fourier级数
$$
\sum_{n \in \mathbb{Z}} \hat{f}(n)\mathrm{e}^{inx}
$$
\par
绝对收敛,并且一致收敛.

\par
(4)我们从(1)开始修改,先列出条件:存在$M>0$,
$$
|f(x)-f(y)|\leqslant M|x-y|^{\alpha}, \alpha>\dfrac{1}{2}
$$
\par
$$
\begin{aligned}
\sum_{n=-\infty}^{+\infty}
4\left|\sin nh\right|^2\left|\hat{f}(n)\right|^2
\leqslant &\dfrac{1}{2\pi}\int_{0}^{2\pi}\left|g_h(x)\right|^2\mathrm{d}x\\
\leqslant &\dfrac{1}{2\pi}\int_{0}^{2\pi}2^{2\alpha}M^2h^{2\alpha}\mathrm{d}x\\
= &2^{2\alpha}M^2h^{2\alpha}
\end{aligned}
$$
\par
于是我们得到:
$$
\sum_{n=-\infty}^{+\infty}
\left|\sin nh\right|^2\left|\hat{f}(n)\right|^2
\leqslant 2^{2\alpha-2}M^2h^{2\alpha}
$$
\par
继续修改有:
$$
\begin{aligned}
\sum_{2^{p-1}<|n|\leqslant 2^p}|\sin nh|^2\left|\hat{f}(n)\right|^2
\leqslant
\sum_{n=-\infty}^{+\infty}
\left|\sin nh\right|^2\left|\hat{f}(n)\right|^2
\leqslant  2^{2\alpha-2}M^2h^{2\alpha}
\end{aligned}
$$
\par
取$h=\dfrac{\pi}{2^{p+1}}$,则
$$
\begin{aligned}
    \sum_{2^{p-1}<|n|\leqslant 2^p}|\sin nh|^2\left|\hat{f}(n)\right|^2
    \leqslant  M^2\dfrac{\pi^{2\alpha}}{2^{2\alpha p+2}}   
    \end{aligned}
$$
\par
而此时$nh \in \left(\dfrac{\pi}{4},\dfrac{\pi}{2}\right]$
\par
则
$$
\begin{aligned}
    \sum_{2^{p-1}<|n|\leqslant 2^p}\dfrac{1}{2}\cdot\left|\hat{f}(n)\right|^2
    \leqslant  M^2\dfrac{\pi^{2\alpha}}{2^{2\alpha p+2}}
    \end{aligned}
$$
\par
整理得到:
$$
\sum_{2^{p-1}<|n|\leqslant 2^p}\left|\hat{f}(n)\right|^2
    \leqslant  M^2\dfrac{\pi^{2\alpha}}{2^{2\alpha p+1}}
$$
\par
我们再次直接估计,
$$
\begin{aligned}
\sum_{2^{p-1}<|n|\leqslant 2^p}\left|\hat{f}(n)\right|
    \leqslant & \left(\sum_{2^{p-1}<|n|\leqslant 
    2^p}\left|\hat{f}(n)\right|^2\right)^{\frac{1}{2}}
    \cdot \left(\sum_{2^{p-1}<|n|\leqslant 2^p}1^2\right)^{\frac{1}{2}}\\
    \leqslant & M\dfrac{\pi^{\alpha}}{2^{\alpha p+\frac{1}{2}}}\cdot 2^{\frac{p}{2}-\frac{1}{2}}\\
    = & \dfrac{M\pi^{\alpha}}{2}{2^{p(\frac{1}{2}-\alpha)}}\\
\end{aligned}
$$
\par
进一步,由于$\dfrac{1}{2}-\alpha<0$:
$$
\begin{aligned}
\sum_{n=-\infty}^{+\infty}\left|\hat{f}(n)\right|
=&\sum_{|n|\leqslant 1}\left|\hat{f}(n)\right|
+\sum_{p\geqslant 1}\sum_{2^{p-1}<|n|\leqslant 2^p}\left|\hat{f}(n)\right|\\
\leqslant & \sum_{|n|\leqslant 1}\left|\hat{f}(n)\right|
+ \sum_{p\geqslant 1}  \dfrac{M\pi^{\alpha}}{2}{2^{p(\frac{1}{2}-\alpha)}}\\
=& \sum_{|n|\leqslant 1}\left|\hat{f}(n)\right|
+ \dfrac{M\pi^{\alpha}}{2}\cdot \dfrac{1}{1-2^{\frac{1}{2}-\alpha}}\\
<&+\infty
\end{aligned}
$$
\par
则其Fourier级数
$$
\sum_{n \in \mathbb{Z}} \hat{f}(n)\mathrm{e}^{inx}
$$
\par
绝对收敛,并且一致收敛.












\end{solution}



\begin{problem}
    Exercise 17
    \par
    已知函数$f$在$[-\pi,\pi]$上单调有界,证明:
    $$
    \hat{f}(n)=O\left(\dfrac{1}{|n|}\right)
    $$
\end{problem}

\begin{solution}
\par
证明: $f$有界,那么存在$M>0$,使得
$$
\left|f\right|\leqslant M
$$
\par
考虑特征函数$\chi_{[a,b]}(x)$的Fourier系数,
$$
\hat{\chi}_{[a,b]}(n)=\dfrac{\mathrm{e}^{-ina}-\mathrm{e}^{-inb}}{2\pi in}
$$
\par
则:
$$
\left|\hat{\chi}_{[a,b]}(n)\right|\leqslant \dfrac{1}{\pi |n|}
$$
\par
则$\hat{\chi}(n)= O\left(\dfrac{1}{|n|}\right)$
\par
现在考虑任意$[-\pi,\pi]$上的分段阶梯函数
$$
h(x)=\sum_{k=1}^{N}\alpha_k\chi_{[a_k,a_{k+1}]}(x)
$$
\par
其中$-\pi=a_1<a_2<\cdots<a_N<a_{N+1}=\pi$且
$-M\leqslant \alpha_1\leqslant \cdots \leqslant \alpha_N\leqslant M$
其Fourier系数
$$
\begin{aligned}
\hat{h}(n)=&\sum_{k=1}^{N}\alpha_k\cdot \dfrac{\mathrm{e}^{-ina_k}-\mathrm{e}^{-ina_{k+1}}}{2\pi in}\\
=&\dfrac{1}{2\pi in}\sum_{k=1}^{N}\alpha_k \cdot (\mathrm{e}^{-ina_k}-\mathrm{e}^{-ina_{k+1}})
\end{aligned}
$$
\par
Abel求和就有:
$$
\begin{aligned}
&\hat{h}(n)\\
=&\dfrac{1}{2\pi in}\sum_{k=1}^{N}\alpha_k \cdot (\mathrm{e}^{-ina_k}-\mathrm{e}^{-ina_{k+1}})\\
=&\dfrac{1}{2\pi in}\left(\alpha_N\left(\mathrm{e}^{-ina_1}-\mathrm{e}^{-ina_{N+1}}\right)
+\sum_{k=1}^{N-1}(\alpha_k-\alpha_{k+1})
\left(\mathrm{e}^{-ina_1}-\mathrm{e}^{-ina_{k+1}}\right)\right)
\end{aligned}
$$
\par
则
$$
\begin{aligned}
\left|\hat{h}(n)\right|
\leqslant & \dfrac{1}{2\pi |n|}\left(2|\alpha_N|
+2\sum_{k=1}^{N-1}(\alpha_{k+1}-\alpha_{k})\right)\\
= & \dfrac{1}{2\pi |n|}\left(2|\alpha_N|
+2(\alpha_{N}-\alpha_{1})\right)\\
\leqslant & \dfrac{6M}{2\pi |n|}
\end{aligned}
$$
\par
则
$$
\hat{h}(n)=O\left(\dfrac{1}{|n|}\right)
$$
\par
且注意到这是关于$N$,$\alpha_i,a_i$一致的.
\par
$f$是单调有界函数,对任意$\varepsilon>0$,则存在阶梯函数$h'(x)$使得,
$$
\dfrac{1}{2\pi}\int_{-\pi}^{\pi}|f(x)-h'(x)|\mathrm{d}x<\varepsilon
$$
\par
则
$$
\begin{aligned}
\left|\hat{f}(n)\right|
\leqslant & \left|\hat{h}'(n)\right|+\left|\hat{f}(n)-\hat{h}'(n)\right|\\
=& O\left(\dfrac{1}{|n|}\right)+\left|\dfrac{1}{2\pi}\int_{-\pi}^{\pi}\left(f(x)-h'(x)\right)\mathrm{e}^{-inx}\mathrm{d}x\right|\\
\leqslant & O\left(\dfrac{1}{|n|}\right) +\dfrac{1}{2\pi}\int_{-\pi}^{\pi}\left|f(x)-h'(x)\right|\mathrm{d}x\\
=& O\left(\dfrac{1}{|n|}\right)+\varepsilon
\end{aligned}
$$
\par
根据$O\left(\dfrac{1}{|n|}\right)$关于$N,\alpha_I,a_i$的一致性和$\varepsilon>0$的任意性,我们有:
$$
\left|\hat{f}(n)\right|=O\left(\dfrac{1}{|n|}\right)
$$
\end{solution}



\begin{problem}
    Exercise 18
    \par
    通过证明对于每个收敛到$0$的非负数列$\{\varepsilon_n\}$,存在连续函数$f$,使得
    存在可数无穷多个$n$使得:
    $$
    \left|\hat{f}(n)\right| \geqslant \varepsilon_n
    $$
    \par
    以此证明: 连续函数的Fourier系数级数可以以任意慢的速度收敛到$0$.
\end{problem}

\begin{solution}
\par
证明:
\par
已知对任意$\{\varepsilon_n\}$,任意$k\in \mathbb{N}$,存在$n_k\in \mathbb{N}$且$n_k>n_{k-1}$,
使得:
$$
|\varepsilon_{n_k}|<\dfrac{1}{k^2}
$$
\par
下定义:
$$
f(x)=\sum_{k=1}^{\infty}\varepsilon_{n_k}\mathrm{e}^{in_kx}
$$
\par
根据Weierstrass判别法:
$$
\sum_{k=1}^{\infty}|\varepsilon_n|<\sum_{k=1}^{\infty}\dfrac{1}{k^2}<\infty
$$
\par
此函数是良定义的,由于一致收敛,是连续函数,且满足:
$$
\hat{f}(n_k)=\varepsilon_{n_k}\geqslant \varepsilon_{n_k}, \forall k \in \mathbb{N}
$$
\par
原命题得证.
\end{solution}



\begin{problem}
    Exercise 19
    \par
    另一种证明$\sum_{0<|n|\leqslant N}\dfrac{\mathrm{e}^{inx}}{n}$关于$N$
    和$x\in[-\pi,\pi]$一致有界的方法如下:
    \par
    利用事实:
    $$
    \dfrac{1}{2i}\sum_{0<|n|\leqslant N}\dfrac{\mathrm{e}^{inx}}{n}
    =\sum_{n=1}^{N}\dfrac{\sin nx}{n}
    =\dfrac{1}{2}\int_{0}^{x}\left(D_N(t)-1\right)\mathrm{d}t
    $$
    \par
    其中$D_N(t)$是Dirichlet核, 并利用Exercise 12结论:
    $$
    \int_{0}^{\infty}\dfrac{\sin x}{x}\mathrm{d}x=\dfrac{\pi}{2}<\infty
    $$
\end{problem}

\begin{solution}
\par
证明:
\par
$$
\begin{aligned}
    &\left|\dfrac{1}{2}\int_{0}^{x}(D_N(t)-1)\mathrm{d}t\right|\\
    \leqslant & \dfrac{1}{2}\left|\int_{0}^{x}D_N(t)\mathrm{d}t\right|
    +\dfrac{|x|}{2}\\
    \leqslant & \dfrac{\pi}{2}+\dfrac{1}{2}\left|\int_{0}^{x}D_N(t)\mathrm{d}t\right|\\
    = &\dfrac{\pi}{2}+\dfrac{1}{2}\left|\int_{0}^{x}\dfrac{\sin (N+\frac{1}{2})t}{\sin \frac{1}{2}t}\mathrm{d}t\right|\\
    =&\dfrac{\pi}{2}+\dfrac{1}{2}\left|\int_{0}^{x}\left(
        \dfrac{\sin (N+\frac{1}{2})t}{\sin \frac{1}{2}t}
    -\dfrac{\sin (N+\frac{1}{2})t}{\frac{1}{2}t}+
    \dfrac{\sin (N+\frac{1}{2})t}{\frac{1}{2}t}\right)\mathrm{d}t\right|\\
    \leqslant &   \dfrac{\pi}{2}+\dfrac{1}{2}\left|\int_{0}^{|x|}
        \sin \left(N+\frac{1}{2}\right)t\left( \dfrac{1}{\sin \frac{1}{2}t}
    -\dfrac{1}{\frac{1}{2}t}\right)\mathrm{d}t\right|
    +\dfrac{1}{2}\left| \int_{0}^{|x|}
    \dfrac{\sin (N+\frac{1}{2})t}{\frac{1}{2}t}\mathrm{d}t\right|\\
\end{aligned}
$$
\par
估计第一部分,根据
$$
\dfrac{t}{2}-\dfrac{t^3}{48}\leqslant \sin \frac{t}{2}\leqslant \dfrac{t}{2}, \quad \forall t>0
$$
\par
利用上述不等式通分放缩得到:
$$
0\leqslant \left( \dfrac{1}{\sin \frac{1}{2}t}
-\dfrac{1}{\frac{1}{2}t}\right) \leqslant \dfrac{2t}{24-t^2}
$$
\par
$$
\begin{aligned}
&\dfrac{1}{2}\left|\int_{0}^{|x|}
\sin \left(N+\frac{1}{2}\right)t\left( \dfrac{1}{\sin \frac{1}{2}t}
-\dfrac{1}{\frac{1}{2}t}\right)\mathrm{d}t\right|\\
\leqslant & \dfrac{1}{2}\int_{0}^{|x|} \left( \dfrac{1}{\sin \frac{1}{2}t}
-\dfrac{1}{\frac{1}{2}t}\right) \mathrm{d}t\\
\leqslant & \dfrac{1}{2}\int_{0}^{|x|}\dfrac{2t}{24-t^2} \mathrm{d}t\\
=& \dfrac{1}{2}\ln\left(\dfrac{24}{24-|x|^2}\right)\\
\leqslant & \dfrac{1}{2}\cdot \dfrac{|x|^2}{24-|x|^2}\\
\leqslant & \dfrac{\pi^2}{48-2\pi^2}
\end{aligned}
$$
\par
估计第二部分,
$$
\begin{aligned}
&\dfrac{1}{2}\left| \int_{0}^{|x|}
    \dfrac{\sin (N+\frac{1}{2})t}{\frac{1}{2}t}\mathrm{d}t\right|\\
=& \dfrac{1}{2}\left| \int_{0}^{|x|}
\dfrac{\sin (N+\frac{1}{2})t}{\frac{1}{2}t}\mathrm{d}t\right|\\
=&\left| \int_{0}^{(N+\frac{1}{2})|x|}
\dfrac{\sin u}{u}\mathrm{d}u\right|\\
\leqslant & \int_{0}^{\pi}\dfrac{\sin u}{u}\mathrm{d}u\\
=& C
\end{aligned}
$$
\par
$C$是与$N,x$无关常数.
\par
综上所述,
$$
\left|\dfrac{1}{2}\int_{0}^{x}(D_N(t)-1)\mathrm{d}t\right|\leqslant 
\dfrac{\pi}{2}+\dfrac{\pi^2}{48-2\pi^2}+C
$$
\par
原命题得证.
\end{solution}



\begin{problem}
    Exercise 20
    \par
    设$f(x)$是锯齿函数$f(x)=(\pi-x)/2,x\in(0,2\pi)$,且
    $f(0)=0$可延拓为$\mathbb{R}$上的周期函数.
    \par
    其Fourier系数是
    $$
    f(x)\sim \dfrac{1}{2i}\sum_{|n|\ne 0}\dfrac{\mathrm{e}^{inx}}{n}
    =\sum_{n=1}^{\infty}\dfrac{\sin nx}{n}
    $$
    \par
    $f$有跳跃间断点在$x=0$, $f(0^+)=\dfrac{\pi}{2}$, $f(0^-)=-\dfrac{\pi}{2}$.
    \par
    证明:
    $$
    \lim_{N \to \infty}\max_{0<x\leqslant \frac{\pi}{N}}\{S_N(f)(x)-f(x)\}=
    \int_{0}^{\pi}\dfrac{\sin t}{t}\mathrm{d}t-\dfrac{\pi}{2}\approx 0.09\pi
    $$
\end{problem}

\begin{solution}
\par
证明:
\par
有
$$
S_N(f)(x)=\dfrac{1}{2}\int_{0}^{x}(D_N(t)-1)\mathrm{d}t
$$
\par
则求导得:
$$
\begin{aligned}
&\dfrac{\mathrm{d}}{\mathrm{d}x}\left(S_N(f)(x)-f(x)\right)\\
=&\dfrac{1}{2}D_N(x)\\
=&\dfrac{\sin(N+\frac{1}{2})x}{2\sin \frac{x}{2}}\\
\end{aligned}
$$
\par
极大值点$x=\dfrac{(2k+1)\pi}{N+\frac{1}{2}}$,极小值点$x=\dfrac{2k\pi}{N+\frac{1}{2}}$, $k\geqslant 1$
\par
则代入$\left(0,\dfrac{\pi}{N}\right)$中的极大值点和最大值点$x_0=\dfrac{\pi}{N+\frac{1}{2}}$得到:
$$
\begin{aligned}
    &\max_{0<x\leqslant \frac{\pi}{N}}\{S_N(f)(x)-f(x)\}\\
=&S_N(f)(x_0)-f(x_0)\\
=& \sum_{k=1}^{N} \dfrac{\sin \left(k\cdot \dfrac{\pi}{N+\frac{1}{2}}\right)}{k}
-\dfrac{\pi}{2} +\dfrac{\pi}{2N+1}\\
=& \sum_{k=1}^{N} \dfrac{\sin \left(k\cdot \dfrac{\pi}{N+\frac{1}{2}}\right)}{k
\cdot \dfrac{\pi}{N+\frac{1}{2}}}\cdot \dfrac{\pi}{N}\cdot \dfrac{N}{N+\frac{1}{2}}
-\dfrac{\pi}{2} +O\left(\dfrac{1}{N}\right)
\end{aligned}
$$
\par
$$
\lim_{N \to \infty}\max_{0<x\leqslant \frac{\pi}{N}}\{S_N(f)(x)-f(x)\}=\int_{0}^{\pi}\dfrac{\sin x}{x}\mathrm{d}x-\dfrac{\pi}{2}\approx 0.09\pi
$$
\end{solution}





\begin{problem}
     Problem 2
     \par
     我们熟知函数族$\{\mathrm{e}^{inx}\}$是在Riemann可积函数空间$\mathcal{R}$中正交的且是完备的(
        表现在Fourier级数依范数收敛到$f$
     ).
     \par
     我们现在考虑具有相同性质的其他一族函数.
     \par
     在$[-1,1]$上定义,
     $$
     L_n(x)=\dfrac{\mathrm{d}^n}{\mathrm{d}x^n}(x^2-1)^n, \quad n=0,1,2,\cdots
     $$
     \par
     则$L_n$是$n$次多项式, 称为$n$阶Legendre多项式.
     \par
     (1)证明:
     如果$f$在$[-1,1]$上无穷阶可导, 有:
     $$
       \int_{-1}^{1}L_n(x)f(x)\mathrm{d}x=(-1)^n\int_{-1}^{1} (x^2-1)^nf^{(n)}(x)\mathrm{d}x
     $$
    \par
    特殊的, 证明:$L_n$与$x^m,m<n$正交,因此$\{L_n\}_{n=0}^{\infty}$是一个正交函数族.
    \par
    (2)证明:
    $$
    \|L_n\|^2=\int_{-1}^{1}|L_n(x)|^2
\mathrm{d}x=\dfrac{(n!)^22^{2n+1}}{2n+1}
    $$
    \par
    (3)证明:任何正交于$1,x,x^2,\cdots,x^{n-1}$的$n$次多项式是$L_n$的常数倍.
    \par
    (4)设$\mathcal{L}_n=\dfrac{L_n}{\|L_n\|}$,即为标准化的Legendre多项式.
    \par
    证明:$\{\mathcal{L}_n\}$是对$\{1,x,x^2,\cdots,x^n,\cdots\}$Schmidt正交化得到的,
    并且每个$[-1,1]$上的Riemann可积函数$f$都有Legendre 展开:
    $$
    \sum_{n=0}^{\infty}\left \langle   f,\mathcal{L}_n\right \rangle\mathcal{L}_n
    $$
\end{problem}

\begin{solution}
\par
先定义内积
$$
\left \langle f, g\right \rangle=\int_{-1}^{1}f(x)\overline{g(x)}\mathrm{d}x
$$

\par
(1)证明:
\par
利用Leibniz法则,容易验证:
$$
\dfrac{\mathrm{d}^m}{\mathrm{d}x^m}(x^2-1)^n \bigg|_{x=-1,1}=0
$$
\par
已知$f\in C^{\infty}$,利用分部积分直接计算:
$$
\begin{aligned}
&\int_{-1}^{1}L_n(x)f(x)\mathrm{d}x\\
=&\dfrac{\mathrm{d}^{n-1}}{\mathrm{d}x^{n-1}}(x^2-1)^nf(x)\bigg|^1_{-1}
-\int_{-1}^{1}\dfrac{\mathrm{d}^{n-1}}{\mathrm{d}x^{n-1}}(x^2-1)^nf'(x)\mathrm{d}x\\
=&-\int_{-1}^{1}\dfrac{\mathrm{d}^{n-1}}{\mathrm{d}x^{n-1}}(x^2-1)^nf'(x)\mathrm{d}x\\
=&(-1)^2\int_{-1}^{1}\dfrac{\mathrm{d}^{n-2}}{\mathrm{d}x^{n-2}}(x^2-1)^nf^{(2)}(x)\mathrm{d}x\\
=&\cdots\\
=&(-1)^n\int_{-1}^{1}(x^2-1)^nf^{(n)}(x)\mathrm{d}x\\
\end{aligned}
$$
\par
当$m<n$,$(x^m)^{(n)}=0$
$$
\int_{-1}^{1}L_n(x)x^m\mathrm{d}x=(-1)^n\int_{-1}^{1}(x^2-1)^n\cdot 0\mathrm{d}x=0
$$
\par
则$L_n$与$x^m,m<n$正交,因为$L_n(x)$是$n$次多项式,则$\{L_n\}_{n=0}^{\infty}$是一个正交函数族.
\par
(2)证明:

$$
\begin{aligned}
\|L_n\|^2
=&\int_{-1}^{1}|L_n(x)|^2\mathrm{d}x\\
=&(-1)^n\int_{-1}^{1}(x^2-1)^n\dfrac{\mathrm{d}^{2n}}{\mathrm{d}x^{2n}}(x^2-1)^n\mathrm{d}x\\
=&(-1)^n(2n)!\int_{-1}^{1}(x^2-1)^n\mathrm{d}x\\
=&(1)^n(2n)!2\cdot\int_{0}^{1}(x^2-1)^n\mathrm{d}x\\
=&(-1)^n(2n)!2\cdot\int_{0}^{\frac{\pi}{2}}(-1)^n\sin^{2n}\theta \sin \theta \mathrm{d}\theta\\
=&2(2n)!\cdot \dfrac{(2n)!!}{(2n+1)!!}\\
=&\dfrac{(n!)^22^{2n+1}}{2n+1}\\
\end{aligned}
$$
\par
(3)
证明:
由于次数小于等于$n$的多项式形成一个$n+1$维线性空间,而$\{L_0,L_1,\cdots,L_n\}$恰好是
$n+1$个线性无关向量,则为其一组正交基.
\par
设满足对$1,x,\cdots,x^{n-1}$都正交的$n$次多项式为:
$$
p_n(x)=a_nL_n(x)+\cdots+a_1L_1(x)+a_0L_0(x)
$$
\par
那么$p_n$对于$1,x,\cdots,x^{n-1}$的线性组合$L_0,L_1,\cdots,L_{n-1}$也正交.
\par
两边做内积就有:
$$
0=\left \langle p_n, L_m\right \rangle=a_m\|L_m\|^2, m=0,1,\cdots,n-1
$$
\par
又$\|L_m\|\ne 0$,则$a_m=0$,于是:
$$
p_n(x)=a_nL_n(x)
$$
\par
原命题成立.
\par
(4)证明:只需要说明当$span\{1,x,\cdots,x^k\}=span\{\mathcal{E}_0,\mathcal{E}_1,\cdots,\mathcal{E}_k\}$时,
有:$$
\mathcal{E}_{k+1}=x^{k+1}-\sum_{i=0}^{k}\left \langle x^{k+1}, \mathcal{E}_i(x)\right \rangle\mathcal{E}_i(x)
$$
\par
则$\mathcal{E}_{k+1}$正交于$\{\mathcal{E}_0,\mathcal{E}_1,\cdots,\mathcal{E}_k\}$,且:
$$
span\{1,x,\cdots,x^k\}=span\{\mathcal{E}_0,\mathcal{E}_1,\cdots,\mathcal{E}_k\}
$$
\par
那么根据$(3)$知道:
$$
\mathcal{E}_{k+1}=\lambda \mathcal{L}_{k+1}
$$
\par
且容易知道:$\lambda>0$.
\par
又容易知道标准化后,则$\{\mathcal{L}_n\}$是对$\{1,x,x^2,\cdots,x^n,\cdots\}$
Gram-Schmidt标准正交化得到的一组标准正交基.
\par
我们继续证明之前,先约定好记号:
\par
范数:
$$
\|f\|=\left(\int_{-1}^{1}|f(x)|^2\mathrm{d}x\right)^{\frac{1}{2}}
$$
\par
内积:
$$
\left \langle \mathcal{L}_m,\mathcal{L}_n \right \rangle=\delta_{mn}
$$
\par
其中$\delta_{mn}$是Kronecker记号.
\par
$f\in \mathcal{R}$时,我们仿照一般的Fourier级数定义,
Legendre系数:
$$
a_n=\left \langle f, \mathcal{L}_n \right \rangle, n \in \mathbb{N}
$$
\par
定义前$N$项和:
$$
S_N(f)(x)=\sum_{i=0}^{N}a_i\mathcal{L}_i(x)
$$

\par
容易验证:
$$
f=f-S_N(f)(x)+S_N(f)(x)
$$
\par
且$f-S_N(f)(x)$与$S_N(f)(x)$正交.
\par
于是有:
$$
\|f\|^2=\|f-S_N(f)\|^2+\sum_{i=0}^{n}|a_i|^2
$$
\par
我们还有最佳逼近引理:
$$
\|f-S_N(f)(x)\|\leqslant\left\|f-\sum_{i=0}^{N}c_i\mathcal{L}_n(x)\right\|
$$
\par
我们继续仿照一般的均方收敛的证明:
\par
若$f(x)$是$[-1,1]$上的连续函数,根据Weierstrass第一逼近定理:
\par
任意$\varepsilon>0$,存在次数为$M$的多项式$P(x)$使得:
$$
\sup|f(x)-P(x)|<\varepsilon
$$
\par
则:
$$
\|f(x)-S_N(f)(x)\|\leqslant \|f(x)-P(x)\|\lesssim \varepsilon
$$
\par
其中$N\geqslant M$
\par
若$f(x)$只是Riemann可积,那么存在$\{g_n(x)\}_{n=1}^\infty$,满足:
$$
\sup|g_n(x)|\leqslant \sup |f(x)|
$$
\par
且
任意$\varepsilon>0$,存在$g_n(x)$
$$
\int_{-1}^{1}|f(x)-g_n(x)|\mathrm{d}x<\varepsilon^2 
$$
\par
那么计算:
$$
\|f(x)-g_n(x)\|\lesssim \varepsilon
$$
\par
再用Werierstrass第一逼近定理:
\par
存在多项式$Q(x)$,使得:
$$
\|g_n(x)-Q(x)\|<\varepsilon
$$
\par
则:
$$
\begin{aligned}
&\|f(x)-S_N(f)(x)\|\\
\leqslant &\|f(x)-Q(x)\| \\
 \leqslant &\|f(x)-g_n(x)\|+\|g_n(x)-Q(x)\|\\
 \lesssim &\varepsilon
\end{aligned}
$$
\par
则均方收敛意义下:
$$
f \overset{L^2}{\longrightarrow}\sum_{n=0}^{\infty}
\left \langle f, \mathcal{L}_n \right \rangle \mathcal{L}_n
$$
\end{solution}


\begin{problem}
    Problem 3
    \par
    设$\alpha\in \mathbb{C} \setminus \mathbb{Z}$.
    \par
    (1)计算以$2\pi$为周期的函数$f(x)=\cos \alpha x, x\in[-\pi,\pi]$的Fourier级数.
    \par
    (2)证明:
    $$ 
    \sum_{n=1}^{\infty}\dfrac{1}{n^2-\alpha^2}=\dfrac{1}{2\alpha^2}-\dfrac{\pi}{2\alpha \tan(\alpha \pi)}
    $$
    \par
    对于所有$u\in \mathbb{C}\setminus \pi \mathbb{Z}$,
    $$
   \cot u=\dfrac{1}{u}+2\sum_{n=1}^{\infty}\dfrac{u}{u^2-n^2\pi^2.}
    $$
    \par
    (3)证明: 对任意$\alpha \in \mathbb{C}\setminus \mathbb{Z}$,有
    $$
    \dfrac{\alpha \pi}{\sin (\alpha \pi)}=1+2\alpha^2\sum_{n=1}^{\infty}\dfrac{(-1)^{n-1}}{n^2-\alpha^2}.
    $$
    \par
    (4)对于$0<\alpha<1$,证明:
    $$
      \int_{0}^{\infty}\dfrac{t^{\alpha-1}}  {1+t}\mathrm{d}t=\dfrac{\pi}{\sin (\alpha \pi) }
    $$
\end{problem}

\begin{solution}
\par
(1)直接计算:
$$
\begin{aligned}
\hat{f}(n)=&\dfrac{1}{2\pi}\int_{-\pi}^{\pi}\cos \alpha x \mathrm{e}^{-inx}\mathrm{d}x\\
=&\dfrac{1}{2\pi}\int_{-\pi}^{\pi}\cos \alpha x \cos nx\mathrm{d}x\\
=&\dfrac{1}{4\pi}\int_{-\pi}^{\pi}[\cos (\alpha +n)x +\cos (\alpha +n)x]\mathrm{d}x\\
=&\dfrac{1}{4\pi}\left(\dfrac{\sin (\alpha +n)x}{\alpha+n} +\dfrac{\sin (\alpha -n)x}{\alpha-n}\right)\bigg|^\pi_{-\pi}\\
=&\dfrac{(-1)^n\alpha}{\pi}\dfrac{\sin \alpha \pi}{\alpha^2-n^2}\\
\end{aligned}
$$
\par
(2)证明: 由(1): $\hat{f}(n)=O\left(\dfrac{1}{|n|^2}\right)$
\par
则Fourier级数一致收敛,即:
$$
\cos \alpha x =\sum_{n=-\infty}^{\infty}\dfrac{(-1)^n\alpha}{\pi}\dfrac{\sin \alpha \pi}{\alpha^2-n^2}\mathrm{e}^{inx}
$$
\par
令$x=\pi$,则:
$$
\cos \alpha \pi=\sum_{n=-\infty}^{\infty}\dfrac{\alpha}{\pi}\dfrac{\sin \alpha \pi}{\alpha^2-n^2}
$$
\par
整理得到:
$$
\sum_{n=1}^{\infty}\dfrac{1}{n^2-\alpha^2}=\dfrac{1}{2\alpha^2}-\dfrac{\pi}{2\alpha\tan(\alpha \pi)}
$$
\par
令$u=\alpha \pi \notin \pi \mathbb{Z}$,
\par
代入上式整理:
\par
$$
   \cot u=\dfrac{1}{u}+2\sum_{n=1}^{\infty}\dfrac{u}{u^2-n^2\pi^2.}
$$
\par
(3)
$$
\cos \alpha x =\sum_{n=-\infty}^{\infty}\dfrac{(-1)^n\alpha}{\pi}\dfrac{\sin \alpha \pi}{\alpha^2-n^2}\mathrm{e}^{inx}
$$
\par
令$x=0$,即:
$$
1=\sum_{n=-\infty}^{\infty}\dfrac{(-1)^n\alpha}{\pi}\dfrac{\sin \alpha \pi}{\alpha^2-n^2}
$$
\par
整理得到:
$$
\dfrac{\alpha \pi}{\sin \alpha \pi}=1+2\alpha^2\sum_{n=1}^{\infty}\dfrac{(-1)^{n-1}}{n^2-\alpha^2}
$$
\par
(4)证明:
\par
首先:
$$
\begin{aligned}
\int_{0}^{\infty}\dfrac{t^{\alpha-1}}{1+t}\mathrm{d}t
=&\int_{0}^{1}\dfrac{t^{\alpha-1}}{1+t}\mathrm{d}t
+\int_{1}^{\infty}\dfrac{t^{\alpha-1}}{1+t}\mathrm{d}t\\
=&\int_{0}^{1}\dfrac{t^{\alpha-1}+t^{(1-\alpha)-1}}{1+t}\mathrm{d}t
\end{aligned}
$$
\par
根据Abel定理:
$$
\begin{aligned}
&\int_{0}^{1}\dfrac{t^{\alpha-1}}{1+t}\mathrm{d}t\\
=&\lim_{x \to 1^-}\int_{0}^{x}\dfrac{t^{\alpha-1}}{1+t}\mathrm{d}t\\
=&\lim_{x\to 1^-}\int_{0}^{x}\sum_{n=0}^{\infty}(-1)^nt^{n+\alpha-1}\mathrm{d}t\\
=&\lim_{x\to 1^-}\sum_{n=0}^{\infty}\int_{0}^{x}(-1)^nt^{n+\alpha-1}\mathrm{d}t\\
=&\lim_{x\to 1^-}\sum_{n=0}^{\infty}\dfrac{(-1)^nx^{n+\alpha}}{n+\alpha}\\
=&\sum_{n=0}^{\infty}\dfrac{(-1)^n}{n+\alpha}
\end{aligned}
$$
\par
上述计算中,其中用到
广义Newton-Leibniz公式,以及$\dfrac{t^{\alpha-1}}{1+t}$在$(0,1]$上的连续性,以及
$$
\int_{0}^{1}\dfrac{t^{\alpha-1}}{1+t}\mathrm{d}t
$$
\par
此瑕积分收敛,因为
$$\lim_{t \to 0^+}=t^{1-\alpha}\cdot\dfrac{t^{\alpha-1}}{1+t}=1
$$
\par
且$0<1-\alpha<1$
\par
$|t|<1$时,
$$
\dfrac{1}{1+t}=\sum_{n=0}^{\infty}(-t)^n
$$
\par
用到$$
\sum_{n=0}^{\infty}(-1)^nt^{n+\alpha-1}
$$
\par
在$[0,x](x<1)$上绝对收敛且一致收敛,于是可以逐项积分.
\par
$$
\sum_{n=0}^{\infty}\dfrac{(-1)^n}{n+\alpha}
$$
\par
根据Dirichlet判别法,级数收敛,则可以推出级数的Abel和收敛于同一值.于是可以交换极限号和求和号.
\par
同理可得:
$$
\int_{0}^{1}\dfrac{t^{(1-\alpha)-1}}{1+t}\mathrm{d}t
=\sum_{n=0}^{\infty}\dfrac{(-1)^n}{n+1-\alpha}
$$
\par
于是,
$$
\begin{aligned}
\int_{0}^{\infty}\dfrac{t^{\alpha-1}}{1+t}\mathrm{d}t
=&\int_{0}^{1}\dfrac{t^{\alpha-1}+t^{(1-\alpha)-1}}{1+t}\mathrm{d}t\\
=&\sum_{n=0}^{\infty}\dfrac{(-1)^n}{n+\alpha}+\sum_{n=0}^{\infty}\dfrac{(-1)^n}{n+1-\alpha}\\
=&\dfrac{1}{\alpha}+2\sum_{n=1}^{\infty}\dfrac{(-1)^{n-1}\alpha}{n^2-\alpha^2}\\
=&\dfrac{\pi}{\sin \alpha \pi}
\end{aligned}
$$
\end{solution}




\begin{problem}
    Problem 4
    \par
    此题中, 我们发现了$$
    \sum_{n=1}^{\infty}\dfrac{1}{n^k}
    $$
    \par
    的公式, 其中$k$是正偶数.
    \par
    我们定义Bernoulli 数$B_n$:
    $$
    \dfrac{z}{\mathrm{e}^z-1}=\sum_{n=0}^{\infty}\dfrac{B_n}{n!}z^n
    $$
    \par
    (1)证明:$$
    B_0=1,B_1=-\dfrac{1}{2},B_2=\dfrac{1}{6},B_3=0,B_4=-\dfrac{1}{30},B_5=0
    $$
    \par
    (2)证明:
    对于$n\geqslant 1$,有:
    $$
    B_n=-\dfrac{1}{n+1}\sum_{k=0}^{n-1}C_{n+1}^{k}B_k
    $$
    \par
    (3)改写得到:
    $$
     \dfrac{z}{\mathrm{e}^z-1}=1-\dfrac{z}{2}+\sum_{n=2}^{\infty}\dfrac{B_n}{n!}z^n
    $$
    \par
    证明: 当$n>1$且是奇数时, $B_n=0$,
    并且:
    $$
   z\cot z=1+\sum_{n=1}^{\infty}(-1)^n\dfrac{2^{2n}B_{2n}}{(2n)!}z^{2n}
    $$
    \par
    (4)Zeta函数定义为:
    $$
   \zeta(s)=\sum_{n=1}^{\infty}\dfrac{1}{n^s}, s>1
    $$
    \par
    通过之前结论说明:
    $$
    x\cot x=1-2\sum_{m=1}^{\infty}\dfrac{\zeta (2m)}{\pi^{2\pi}}x^{2m}
    $$
    \par
    (5)综上得到:
    $$
    \zeta(2m)=(-1)^{m+1}\dfrac{2^{2m-1}\pi^{2m}}{(2m)!}B_{2m}
    $$
\end{problem}

\begin{solution}
\par
(1)证明:
直接计算:
$$
\begin{aligned}
1=&\dfrac{\mathrm{e}^z-1}{z}\cdot\sum_{n=0}^{\infty}\dfrac{B_n}{n!}z^n\\
=& \left(\sum_{n=0}^{\infty}\dfrac{z^n}{(n+1)!}\right)\left(\sum_{n=0}^{\infty}\dfrac{B_n}{n!}z^n\right)
\end{aligned}
$$
\par
根据$\sum_{n=0}^{\infty}\dfrac{z^n}{(n+1)!}$绝对收敛且$\sum_{n=0}^{\infty}\dfrac{B_n}{n!}z^n$收敛,
\par
那么Cauchy乘积收敛:
$$
1=\sum_{k=0}^{\infty}\sum_{i=0}^{k}\left(\dfrac{1}{(i+1)!}\cdot \dfrac{B_{k-i}}{(k-i)!}\right)z^k
$$
\par
对比系数即可,
有:
$$
\begin{aligned}
1&=B_0\\
0&=B_1+\dfrac{B_0}{2!}\\
0&=\dfrac{B_2}{2!}+\dfrac{B_1}{2!}+\dfrac{B_0}{3!}\\
0&=\dfrac{B_3}{3!}+\dfrac{B_2}{(2!)^2}+\dfrac{B_1}{3!}+\dfrac{B_0}{4!}\\
0&=\dfrac{B_4}{4!}+\dfrac{B_3}{2!\cdot 3!}+\dfrac{B_2}{3!\cdot 2!}+\dfrac{B_1}{4!}+\dfrac{B_0}{5!}\\
0&=\dfrac{B_5}{5!}+\dfrac{B_4}{2!\cdot 4!}
+\dfrac{B_3}{3!\cdot 3!}+\dfrac{B_2}{4! \cdot 2!}+\dfrac{B_1}{5!}+\dfrac{B_0}{6!}
\end{aligned}
$$
\par
解得:
$$
B_0=1,B_1=-\dfrac{1}{2},B_2=\dfrac{1}{6},B_3=0,B_4=-\dfrac{1}{30},B_5=0
$$


\par
(2)证明:由(1)对比系数知道:
$n\geqslant 1$时,
$$
\sum_{i=0}^{n}\dfrac{B_{n-i}}{(i+1)!(n-i)!}=0
$$
\par
整理得到:
$$
\begin{aligned}
B_n=&-\sum_{i=1}^{n}\dfrac{n!}{(i+1)!(n-i)!}B_{n-i}\\
=&-\dfrac{1}{n+1}\sum_{k=0}^{n-1}\dfrac{(n+1)!}{(n-k+1)!k!}B_k\\
=&-\dfrac{1}{n+1}C_{n+1}^{k}B_k
\end{aligned}
$$

\par
(3)证明:
\par
有
$$
\begin{aligned}
\dfrac{z}{\mathrm{e}^z-1}+\dfrac{z}{2}=1+\sum_{n=2}^{\infty}\dfrac{B_n}{n!}z^n
\end{aligned}
$$
\par
由于$\dfrac{z}{\mathrm{e}^z-1}+\dfrac{z}{2}=
\dfrac{z}{2}\cdot\dfrac{\mathrm{e}^{\frac{z}{2}}
+\mathrm{e}^{-\frac{z}{2}}}{\mathrm{e}^{\frac{z}{2}}-\mathrm{e}^{-\frac{z}{2}}}$
是偶函数.
\par
则$n>1$且$n$为奇数时,$B_n=0$.
\par
则我们继续推导:
$$
\begin{aligned}
    \frac{z}{2}\cot \frac{z}{2}&=\frac{z}{2}\frac{\cos \frac{z}{2}}{\sin \frac{z}{2}}\\
           &=\frac{iz}{2}\dfrac{{\mathrm{e}^{i\frac{z}{2}}}-\mathrm{e}^{-i\frac{z}{2}}}{\mathrm{e}^{i\frac{z}{2}}-\mathrm{e}^{-i\frac{z}{2}}}\\
           &=1+\sum_{n=1}^{\infty}\dfrac{B_{2n}}{(2n)!}(iz)^{2n}\\
           &=1+\sum_{n=1}^{\infty}(-1)^n\dfrac{B_{2n}}{(2n)!}z^{2n}\\
\end{aligned}
$$
\par
则
$$
z\cot z=1+\sum_{n=1}^{\infty}(-1)^n\dfrac{2^{2n}B_{2n}}{(2n)!}z^{2n}\\
$$


\par
(4)证明:
我们引Problem 3中(2)的结论:
$$
\cot u=\dfrac{1}{u}+2\sum_{n=1}^{\infty}\dfrac{u}{u^2-n^2\pi^2}, \quad u \notin \mathbb{C} \setminus \pi \mathbb{Z}
$$
\par
取$|x|<\pi$
$$
\begin{aligned}
x\cot x=&1+2\sum_{n=1}^{\infty}\dfrac{x^2}{x^2-n^2\pi^2}\\
=&1+2\sum_{n=1}^{\infty}\dfrac{x^2}{x^2-n^2\pi^2}\\
=&1-2\sum_{n=1}^{\infty}\dfrac{\frac{x^2}{n^2\pi^2}}{1-\frac{x^2}{n^2\pi^2}}\\
=&1-2\sum_{n=1}^{\infty}\sum_{j=1}^{\infty}\dfrac{x^{2j}}{n^{2j}\pi^{2j}}\\
\end{aligned}
$$
\par
由于存在$|x|<\delta<\pi$
$$
\begin{aligned}
&\sum_{n=1}^{\infty}\sum_{j=1}^{\infty}\left|\dfrac{x^{2j}}{n^{2j}\pi^{2k}}\right|\\
\leqslant & \sum_{n=1}^{\infty}\sum_{j=1}^{\infty}\dfrac{\delta^{2j}}{n^{2j}\pi^{2j}}\\
=& \sum_{n=1}^{\infty}\dfrac{\delta^2}{n^2\pi^2-\delta^2}\\
<&\infty
\end{aligned}
$$
\par
可以交换求和次序,
$$
\begin{aligned}
x\cot x
=&1-2\sum_{n=1}^{\infty}\sum_{j=1}^{\infty}\dfrac{x^{2j}}{n^{2j}\pi^{2j}}\\
=&1-2\sum_{j=1}^{\infty}\sum_{n=1}^{\infty}\dfrac{x^{2j}}{n^{2j}\pi^{2j}}\\
=&1-2\sum_{j=1}^{\infty}\zeta(2j)\dfrac{x^{2j}}{\pi^{2j}}\\
\end{aligned}
$$
\par
(5)证明:引用(3)和(4)的结论:
$$
1+\sum_{n=1}^{\infty}(-1)^n\dfrac{2^{2n}B_{2n}}{(2n)!}z^{2n}=
1-2\sum_{n=1}^{\infty}\zeta(2n)\dfrac{z^{2n}}{\pi^{2n}}
$$
\par
对比系数得到:
$$
\zeta(2n)=(-1)^{n+1}\dfrac{2^{2n-1}\pi^{2n}}{(2n)!}B_{2n}
$$
\end{solution}




\begin{problem}
    Problem 5
    \par
    定义Bernoulli多项式如下:
    $$
    \dfrac{z\mathrm{e}^{xz}}{\mathrm{e}^z-1}=\sum_{n=0}^{\infty}\dfrac{B_n(x)}{n!}z^n
    $$
    \par
    (1)函数$B_n(x)$是关于$x$的多项式,且证明:
    $$
    B_n(x)=\sum_{k=0}^{n}C_n^kB_kx^{n-k}
    $$
    \par
    以及:
    $$
    \begin{aligned}
    &B_0(x)=1, B_1(x)=x-\dfrac{1}{2}, \\
    &B_2(x)=x^2-x+\dfrac{1}{6}, B_3(x)=x^3-\dfrac{3}{2}x^2+\dfrac{1}{2}x
    \end{aligned}
    $$
    \par
    (2)证明:
    \par
    $n\geqslant 1$,则:
    $$
    B_n(x+1)-B_n(x)=nx^{n-1}
    $$
    \par
    若$n\geqslant 2$,则;
    $$
     B_n(0)=B_n(1)=B_n
    $$
    \par
    (3)定义:$S_m(n)=1^m+2^m+\cdots+(n-1)^m$.
    \par
    证明:
    $$
     (m+1)S_m(n)=B_{m+1}(n)-B_{m+1}
    $$
    \par
    (4)证明: Bernoulli多项式是唯一满足下列性质的多项式:
    $$
    \begin{aligned}
        &(\text{i})B_0(x)=1\\
        &(\text{ii})B_n'(x)=nB_{n-1}(x), n\geqslant 1\\
        &(\text{iii})\int_{0}^{1}B_n(x)\mathrm{d}x=0, n\geqslant 1
    \end{aligned}
    $$
    \par
    证明:由(2)可得:
    $$
   \int_{x}^{x+1}B_n(t)\mathrm{d}t=x^n
    $$
    \par
    (5)计算$B_1(x)$的Fourier级数,得到$0<x<1$时,
    $$
   B_1(x)=x-\dfrac{1}{2}=-\dfrac{1}{\pi}\sum_{k=1}^{\infty}\dfrac{\sin(2\pi kx) }{k}
    $$
    \par
    积分可得:
    $$
    \begin{aligned}
    &B_{2n}(x)=(-1)^{n+1}\dfrac{2(2n)!}{(2\pi)^{2n}}\sum_{k=1}^{\infty}\dfrac{\cos (2\pi kx)}{k^{2n}}\\
    &B_{2n+1}(x)=(-1)^{n+1}\dfrac{2(2n+1)!}{(2\pi)^{2n+1}}\sum_{k=1}^{\infty}\dfrac{\sin(2\pi kx)}{k^{2n+1}}
    \end{aligned}
    $$
    \par
    最后证明: 当$0<x<1$时,
    $$
    B_{n}(x)=-\dfrac{n!}{(2\pi i)^n}\sum_{n \ne 0}\dfrac{\mathrm{e}^{2\pi i k x}}{k^n}
    $$
\end{problem}

\begin{solution}
\par
(1)证明:
利用Cauchy乘积:
$$
\begin{aligned}
&\sum_{n=0}^{\infty}\dfrac{B_n(x)}{n!}z^n\\
=&\left(\sum_{n=0}^{\infty}\dfrac{B_n}{n!}z^n\right)
\left(\sum_{n=0}^{\infty}\dfrac{x^nz^n}{n!}\right)\\
=&\sum_{n=0}^{\infty}\sum_{k=0}^{n}\dfrac{B_k}{k!(n-k)!}x^{n-k}z^n
\end{aligned}
$$
\par
对比系数:
$$
B_n(x)=\sum_{k=0}^{n}\dfrac{n!B_k}{k!(n-k)!}x^{n-k}=\sum_{k=0}^{n}C_n^kB_kx^{n-k}
$$
\par
于是计算:
$$
\begin{aligned}
&B_0(x)=B_0=1.\\
&B_1(x)=B_0x-B_1x=x-\dfrac{1}{2}\\
&B_2(x)=B_0x^2+2B_1x+B_2=x^2-x+\dfrac{1}{6}\\
&B_3(x)=B_0x^3+3B_1x^2+3B_2x+B_3=x^3-\dfrac{3}{2}x^2+\dfrac{1}{2}x
\end{aligned}
$$
\par
(2)证明:
有:
$$
\begin{aligned}
\sum_{n=0}^{\infty}\dfrac{B_n(x+1)}{n!}z^n
=&\dfrac{z\mathrm{e}^{(x+1)z}}{\mathrm{e}^z-1}\\
=&\dfrac{z\mathrm{e^{xz}}}{\mathrm{e}^z-1}+z\mathrm{e}^{xz}\\
=&\sum_{n=0}^{\infty}\dfrac{B_n(x)}{n!}z^n+\sum_{n=0}^{\infty}\dfrac{x^n}{n!}z^{n+1}\\
=&\sum_{n=0}^{\infty}\dfrac{B_n(x)}{n!}z^n+\sum_{n=1}^{\infty}\dfrac{x^{n-1}}{(n-1)!}z^{n}\\
\end{aligned}
$$
\par
对比系数:
$$
B_n(x+1)-B_n(x)=nx^{n-1}, \quad n\geqslant 1
$$
\par
在上式和(1)结果中,令$x=0$,则:
$$
B_n(1)=B_n(0)=B_n,\quad n\geqslant 2
$$
\par
(3)
证明:
引(2),令$n=m+1$,从$x=1$求和道$n-1$
\par
计算有:
$$
\begin{aligned}
(m+1)S_m(n)
&=\sum_{x=1}^{n-1}\left(B_{m+1}(x+1)-B_{m+1}(x)\right)\\
&=B_{m+1}(n)-B_{m+1}(1)\\
&=B_{m+1}(n)-B_{m+1}\\
\end{aligned}
$$




\par
(4)证明:
\par
我们先说明Bernoulli多项式满足三条性质:
\par
(i)是显而易见的计算.
\par
(ii)求导:
$$
\begin{aligned}
\dfrac{\mathrm{d}}{\mathrm{d}x}B_n(x)
=&\dfrac{\mathrm{d}}{\mathrm{d}x}\sum_{k=0}^{n}C_n^kB_kx^{n-k}\\
=&n\sum_{k=0}^{n-1}C_{n-1}^{k}B_kx^{n-1-k}\\
=&nB_{n-1}(x)
\end{aligned}
$$
\par
(iii)根据(ii),
对任意$n\in \mathbb{N}$计算:
$$
\begin{aligned}
\int_{0}^{1}B_n(x)\mathrm{d}x
=&\int_{0}^{1}\dfrac{B_{n+1}'}{n+1}\mathrm{d}x\\
=&\dfrac{B_{n+1}(1)-B_{n+1}(0)}{n+1}\\
=&\dfrac{B_{n+1}-B_{n+1}}{n+1}\\
=&0
\end{aligned}
$$
\par
于是Bernoulli多项式满足这三条性质.
\par
我们采用反证法,假设还有不同的多项式$\{P_n(x)\}$满足三条性质:
\par
那么我们设$Q_n(x)=B_n(x)-P_n(x)$,有:
$$
Q_0=B_0(x)-P_0(x)=1-1=0
$$
\par
并且:
$$
\begin{aligned}
Q_n'(x)
&=B_n'(x)-Q_n'(x)\\
&=n(B_{n-1}(x)-Q_{n-1}(x))\\
&=nQ_{n-1}(x),\quad n\geqslant 1\\
\end{aligned}
$$
\par
那么$$
Q_1(x)=C
$$
\par
其中$C$为常数.
\par
又容易知道:
$$
\int_{0}^{1}Q_n(x)\mathrm{d}x, \quad n\geqslant 1
$$
\par
得到:$$
C=0
$$
\par
则可以归纳法知道:
若$Q_k(x)=0$,
仿照上述过程,$Q_{k+1}(x)=0$.
\par
于是对任意$n\in \mathbb{N}$,
$$
B_n(x)-P_n(x)=Q_n(x)=0
$$
\par
于是与假设矛盾,Bernoulli多项式是唯一满足下列性质的多项式.
\par
对于:
$$
B_n(x+1)-B_n(x)=nx^{n-1},\quad n \geqslant 1
$$
\par
我们继续计算:
$$
\begin{aligned}
    \int_{x}^{x+1}B_n(x)\mathrm{d}x
    =&\int_{x}^{x+1}\dfrac{B'_{n+1}(x)}{n+1}\mathrm{d}x\\
    =&\dfrac{B_{n+1}(x+1)-B_{n+1}(x)}{n+1}\\
    =&\dfrac{(n+1)x^n}{n+1}\\
    =&x^n
\end{aligned}
$$
\par
故所有命题成立.
\par
(5)证明:
计算$B_1(x)=x-\dfrac{1}{2}$的Fourier系数:
$$
\begin{aligned}
\widehat{B_1}(n)&=\int_{0}^{1}\left(x-\dfrac{1}{2}\right)\mathrm{e}^{-in2\pi x}\mathrm{d}x\\
&=-\dfrac{1}{in2\pi}
\end{aligned}
$$
\par
则:
$$
B_1(x)\sim -\dfrac{1}{\pi}\sum_{k=1}^{\infty}\dfrac{\sin (2\pi kx)}{k},\quad 0<x<1
$$
\par
根据Dirichlet判别法:
$$
B_1(x)=x-\dfrac{x}{2}=-\dfrac{2\cdot 1!}{2\pi}\sum_{k=1}^{\infty}\dfrac{\sin(2\pi kx)}{k},\quad 0<x<1
$$
\par
$B_1(x)=x-\dfrac{1}{2}$在$[0,1]$上可积且绝对可积,
那么可以逐项积分:
$$
\int_{0}^{x} B_2'(x)\mathrm{d}x=
\int_{0}^{x} 2B_1(x)\mathrm{d}x
=-\dfrac{2}{\pi}\sum_{k=1}^{\infty}\int_{0}^{x}\dfrac{\sin(2\pi kx)}{k}\mathrm{d}x
$$
\par
则:
$$
B_2(x)=\dfrac{2\cdot 2!}{(2\pi)^2}\sum_{k=1}^{\infty}\dfrac{\cos(2\pi kx)}{k^2}
$$
\par
其中化简时积分相消用到$n>1$时:
$$
B_i=
\begin{cases}
    (-1)^{n+1}\dfrac{2\cdot (2n)!}{(2\pi)^{2n}}\cdot \displaystyle{\sum_{k=1}^{\infty}\frac{1}{k^{2n}}}, &i=2n\\
    0,&i=2n+1
\end{cases}
$$
\par
以此用归纳法,
假设有:
$$
\begin{aligned}
   & B_{2n}(x)=(-1)^{n+1}\dfrac{2(2n)!}{(2\pi)^{2n}}\sum_{k=1}^{\infty}\dfrac{\cos(2\pi kx)}{k^{2n}}\\
   & B_{2n+1}(x)=(-1)^{n+1}\dfrac{2(2n+1)!}{(2\pi)^{2n+1}}\sum_{k=1}^{\infty}\dfrac{\sin (2\pi kx)}{k^{2n+1}}\\
\end{aligned}
$$
\par
则我们计算:
$$
\begin{aligned}
&\int_{0}^{x}B'_{2n+2}(x)\mathrm{d}x\\
=&\int_{0}^{x}(2n+2)B_{2n+1}(x)\mathrm{d}x\\
=&(-1)^{n+1}\dfrac{2(2n+2)!}{(2\pi)^{2n+1}}\sum_{k=1}^{\infty}\int_{0}^{x}\dfrac{\sin(2\pi kx)}{k^{2n+1}}\mathrm{d}x
\end{aligned}
$$
\par
则:
$$
B_{2n+2}(x)=(-1)^{n+2}\dfrac{2\cdot (2n+2)!}{(2\pi)^{2n+2}}\sum_{k=1}^{\infty}\dfrac{\cos(2\pi kx)}{k^{2n+2}}
$$
\par
继续计算有:
$$
\begin{aligned}
&\int_{0}^{x}B'_{2n+3}(x)\mathrm{d}x\\
=&\int_{0}^{x}(2n+3)B_{2n+2}(x)\mathrm{d}x\\
=&(-1)^{n+2}\dfrac{2\cdot (2n+3)!}{(2\pi)^{2n+2}}\sum_{k=1}^{\infty}\int_{0}^{x}\dfrac{\cos(2\pi kx)}{k^{2n+2}}
\end{aligned}
$$
\par
于是,
$$
B_{2n+3}=(-1)^{n+2}\dfrac{2\cdot (2n+3)!}{(2\pi)^{2n+3}}\sum_{k=1}^{\infty}\dfrac{\sin(2\pi kx)}{k^{2n+3}}
$$
\par
归纳结束. 命题成立.
\par
结合之前的结论:
$$
B_n(x)=-\dfrac{n!}{(2\pi i)^n}\sum_{k \ne 0}\dfrac{\mathrm{e}^{2\pi ikx}}{k^2},\quad 0<x<1
$$
\end{solution}















\end{document}