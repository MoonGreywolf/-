\documentclass[12pt, a4paper, oneside]{ctexart}
\usepackage{amsmath, amsthm, amssymb, bm, color, framed, graphicx, hyperref, mathrsfs}

\title{\textbf{Fourier Analysis Appendix 5}}
\author{张浩然}
\date{\today}
\linespread{1.5}
\definecolor{shadecolor}{RGB}{241, 241, 255}
\newcounter{problemname}
\newenvironment{problem}{\begin{shaded}\stepcounter{problemname}\par\noindent\textbf{题目\arabic{problemname}. }}{\end{shaded}\par}
\newenvironment{solution}{\par\noindent\textbf{解答. }}{\par}
\newenvironment{note}{\par\noindent\textbf{题目\arabic{problemname}的注记. }}{\par}

\begin{document}

\maketitle



\begin{problem}
  2020年丘成桐大学生数学竞赛分析与方程 Q2
  \par
  在去年的丘成桐大学生数学竞赛中, Tintin, Haddock,
 Dupont和Dupond挺进决赛分析口试.
 \par
 丘先生让他们计算一个周期为$2\pi$的函数(定义在$\mathbb{R}$上)的Foruier系数:
 $$
 \begin{aligned}
 &F: (0,2\pi)\rightarrow \mathbb{R}\\
 &x\mapsto F(x) =\arctan \left(\dfrac{x}{2\pi}\mathrm{e}^{\sin x}
 +x^{2019}(x-2\pi)+2019\sin x \right) 
 \end{aligned}
 $$
 \par
 以下是他们各自的解, $k\neq 0$:
 \par
 Tintin:
  $\widehat{F}(k)=\dfrac{\cos k\pi}{|k|^{\frac{1}{2}}}+\dfrac{a}{|k|}+\dfrac{b}{|k|^3}$
 \par
Haddock: $\widehat{F}(k)=\dfrac{c}{k^2}+\dfrac{d}{k^4}+\dfrac{e}{k^6}$
\par
Dupont: $\widehat{F}(k)=\dfrac{1}{k}+\dfrac{1}{k^2}+\dfrac{f}{k^3}+\dfrac{g}{k^5}$
 \par
 Dupond: $\widehat{F}(k)=\dfrac{2019\sqrt{-1}}{k}+\dfrac{h_k}{k}$
 \par
 其中$a,b,c,d,e, f, g,h_k(k\in \mathbb{Z})$ 是常数
 ,并且:
 $$
 \sum_{k \in \mathbb{Z}}|h_k|^2<\infty
 $$
 \par
 试问谁的解是正确的?
\end{problem}



\begin{solution}
  \par
  我们先研究函数性态:\par
  显然$F\in C^{\infty}(0,2\pi)$.
  $$
   \lim_{x \to 0^+}F(x)=0, \quad \lim_{x \to 2\pi^-}F(x)=\dfrac{\pi}{4}
  $$
  \par
  $F$有界,则$F\in L^2$.
  \par
  对于Tintin的答案, 若正确,我们利用Parseval恒等式有:
  $$
   \widehat{F}(k)\in l^2
  $$
  \par
  但是$\left|\widehat{F}(k)\right| ^2\sim \dfrac{1}{|k|}$,
   那么$\widehat{F}(k)\notin l^2$,此答案错误.
\par
\quad
\par
\quad
\par
对于Haddock的答案,若正确,我们有:
$$
\sum_{k\in \mathbb{Z}}\left|\widehat{F}(k)\right|
\leqslant \sum_{k\in \mathbb{Z}}\dfrac{|c|}{k^2}
+\sum_{k\in \mathbb{Z}}\dfrac{|d|}{k^4}+
\sum_{k\in \mathbb{Z}}\dfrac{|e|}{k^6}
<+\infty
$$
\par
那么$\sum_{k\in\mathbb{Z} }\widehat{F}(k)\mathrm{e}^{\sqrt{-1}kx}$一致收敛, 
收敛到一个连续函数$G(x)$,其 Fourier系数与$F(x)$相同.
\par
\quad
\par
引理: 
\par
若函数$F,G$的Fourier系数相同, 那么在连续点$x=x_0$处,$F(x_0)=G(x_0)$.
\par
\quad
\par
存在$\varepsilon_0=\dfrac{\pi}{16}$,存在$\delta>0$,
使得当$x_1, x_2\in (2\pi-\delta,2\pi+\delta)$时,
$$
|G(x_1)-G(x_2)|<\dfrac{\pi}{16}
$$
\par
接下来取两点$x_1\in(2\pi-\delta,2\pi),x_2\in (2\pi,2\pi+\delta)$,使之充分靠近$2\pi$,
则有:
$$
|F(x_1)-F(x_2)|\geqslant \dfrac{\pi}{16}
$$
\par
又除了$x=2k\pi, k\in \mathbb{Z}$外处处连续,于是$F(x_1)=G(x_1),F(x_2)=G(x_2)$,
那么此答案错误.
\par
\quad
\par
\quad
\par
对于Dupont的答案,其不满足实值函数的Fourier系数的性质:
$$
\overline{\widehat{F}(k)}=\widehat{F}(-k)
$$
\par
故此答案错误.
\par
\quad
\par
\quad
\par
对于Dupond的答案,若正确,则:
$$
\begin{aligned}
\dfrac{2019\sqrt{-1}}{k}+\dfrac{h_k}{k}=&\widehat{F}(k)\\
=&\dfrac{1}{2\pi}\int_{0}^{2\pi}F(x)\mathrm{e}^{-\sqrt{-1}kx}\mathrm{d}x\\
=&\dfrac{\sqrt{-1}}{8k}+\dfrac{\widehat{F'}(k)}{k}\\
\end{aligned}
$$
\par
其中由$F'(x)\in L^2$,则$h_k,\widehat{F'}(k)\in l^2$,但是$2019\sqrt{-1}-\dfrac{1}{8}\sqrt{-1}\notin l^2$
\par
故其答案错误.

\par
\quad
\par
综上所述,没一个正确,如此成绩,令人汗颜! 














\end{solution}
  

\end{document}