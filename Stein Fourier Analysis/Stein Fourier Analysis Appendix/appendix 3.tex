\documentclass[12pt, a4paper, oneside]{ctexart}
\usepackage{amsmath, amsthm, amssymb, bm, color, framed, graphicx, hyperref, mathrsfs}

\title{\textbf{Fourier Analysis Appendix 3}}
\author{张浩然}
\date{\today}
\linespread{1.5}
\definecolor{shadecolor}{RGB}{241, 241, 255}
\newcounter{problemname}
\newenvironment{problem}{\begin{shaded}\stepcounter{problemname}\par\noindent\textbf{题目\arabic{problemname}. }}{\end{shaded}\par}
\newenvironment{solution}{\par\noindent\textbf{解答. }}{\par}
\newenvironment{note}{\par\noindent\textbf{题目\arabic{problemname}的注记. }}{\par}

\begin{document}

\maketitle



\begin{problem}
 2025年丘成桐领军计划四月考试:
 \par
 已知:
 $$
  f(x)=\sum_{n=0}^{\infty}2^{-n}\cos (nx)
 $$
 \par
 求:
 $$
 \int_{0}^{2\pi}|f(x)|^2\mathrm{d}x
 $$
\end{problem}



\begin{solution}
  \par
  显然此题是要考虑Fourier分析:
  \par
  由于
  $2^{-n}|\cos(nx)|\leqslant 2^{-n}$,而且
  $$
   \sum_{n=1}^{\infty}2^{-n}<\infty
  $$
  \par
  根据Weierstrass判别法,其一致收敛且每项连续,则$f$在$[0,2\pi]$上连续.
  \par
  又计算有:
  $$
  f(x)\sim 1+\sum_{n=1}^{\infty}\dfrac{1}{2^{n+1}}\mathrm{e}^{inx}
  +\sum_{n=1}^{\infty}\dfrac{1}{2^{n+1}}\mathrm{e}^{-inx}
  $$
  \par
  根据Paseval恒等式:
  $$
  \dfrac{1}{2\pi}\int_{0}^{2\pi}|f(x)|^2\mathrm{d}x=1+2\sum_{n=1}^{\infty}\dfrac{1}{2^{2n+2}}
  $$
  \par
  则:
  $$
   \int_{0}^{2\pi}|f(x)|^2\mathrm{d}x=\dfrac{7\pi}{3}
  $$
\end{solution}

\begin{problem}

$$
\int_{0}^{2\pi}(x^2-a\cos x-b\sin x)^2\mathrm{d}x
$$
\par
取得最小值时,$a$的值为?
\end{problem}
\begin{solution}
  \par
  我们考虑$[0,2\pi]$上的函数空间内积:
  $$
  \langle f(x),g(x)\rangle=\dfrac{1}{\pi}\int_{0}^{2\pi}f(x)g(x)\mathrm{d}x
  $$
  \par
  范数:
  $$
  \|f\|=\left(\dfrac{1}{\pi}|f(x)|^2\mathrm{d}x\right)^{\frac{1}{2}}
  $$
  \par
  显然,我们在此内积空间上有最佳逼近引理:
  $$
  \|f(x)-\langle f,\cos x \rangle \cos x-\langle f,\sin x \rangle \sin x\|
  \leqslant \|f(x)-a\cos x-b \sin x \|
  $$
\par
取等时,我们有:
$$
a=\langle f,\cos x \rangle =\dfrac{1}{\pi}\int_{0}^{2\pi}x^2\cos x\mathrm{d}x=4
$$

\end{solution}
\end{document}