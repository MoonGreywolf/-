\documentclass[12pt, a4paper, oneside]{ctexart}
\usepackage{amsmath, amsthm, amssymb, bm, color, framed, graphicx, hyperref, mathrsfs}

\title{\textbf{Fourier Analysis Appendix 1}}
\author{张浩然}
\date{\today}
\linespread{1.5}
\definecolor{shadecolor}{RGB}{241, 241, 255}
\newcounter{problemname}
\newenvironment{problem}{\begin{shaded}\stepcounter{problemname}\par\noindent\textbf{题目\arabic{problemname}. }}{\end{shaded}\par}
\newenvironment{solution}{\par\noindent\textbf{解答. }}{\par}
\newenvironment{note}{\par\noindent\textbf{题目\arabic{problemname}的注记. }}{\par}

\begin{document}

\maketitle



\begin{problem}
  Fourier级数定义
  \par
  解古典方程的半幅Fourier展开: $f \in \mathcal{R}[0,\pi]$
  $$
     a_n=\frac{2}{\pi}\int_{0}^{\pi}f(x)\sin x \mathrm{d}x
  $$
  \par
  一般的Fourier展开:
  $$
  \hat{f}(n)=\frac{1}{2\pi}\int_{-\pi}^{\pi}f(\theta)\mathrm{e}^{-in\theta}\mathrm{d}\theta
  $$
\end{problem}



\begin{problem} 
  ODE参数为什么这么取值:
\par
考虑Dirichlet问题,
在矩形区域$R=\{(x,y): 0\leqslant x \leqslant \pi, 0 \leqslant y \leqslant 1\}$
中有
$$
\begin{aligned}
\begin{cases}
  \Delta u=0\\
  u(x,0)=f_0(x)\\
  u(x,1)=f_1(x)\\
  u(0,y)=0\\
  u(\pi,y)=0\\
\end{cases}
\end{aligned}
$$
\par
其中$f_0,f_1$是确定解的初值.


\end{problem}

\begin{solution}
\par
分离变量法,就设$u(x,y)=F(x)G(y)$,
代入$$
\Delta u=0
$$
即$$
\dfrac{F''(x)}{F(x)}=-\dfrac{G''(y)}{G(y)}=\lambda
$$
\par
这是因为左右变量独立,求导可以看出会等于常数$\lambda$.
则
$$
\begin{aligned}
  \begin{cases}
    F''(x)-\lambda F(x)=0\\
    G''(y)+\lambda G(y)=0
  \end{cases}
\end{aligned}
$$
\par
接下来讨论我们需要的解的$\lambda$范围,我们不考虑任何平凡解:
容易排除掉其他情况,得到
$\lambda<0$
\par
解得:$$
\begin{aligned}
&F(x)=C_{1,k}\cos \sqrt{-\lambda}x +C_{2,k} \sin \sqrt{-\lambda}x\\
&G(y)=C_{3,k}\mathrm{e}^{\sqrt{-\lambda}y}+C_{4,k}\mathrm{e}^{-\sqrt{-\lambda}y}
\end{aligned}
$$
\par
又根据边值条件:
$u(0,y)=0, u(\pi,y)=0$
$$
\begin{aligned}
\begin{cases}
C_{1,k}=0\\
C_{1,k}\cos \sqrt{-\lambda}\pi +C_{2,k}\sin \sqrt{-\lambda}\pi=0
\end{cases}
\end{aligned}
$$
\par
那么,$C_{2,k}\sin \sqrt{-\lambda}\pi=0$,我们想要非平凡解,肯定不能让$C_{2,k}=0$,
于是只能$\sin \sqrt{-\lambda}\pi=0$即$\sqrt{-\lambda}\pi=k\pi, \lambda=-k^2, k \in \mathbb{Z}$.

\end{solution}


\begin{problem}
  中科大2012年PDE试题
  \par
  $\varphi \in C_c^{\infty}(0,1)$,即在$(0,1)$上光滑,且具有紧支集,求下列初边值问题:
  $$
  \begin{aligned}
    \begin{cases}
  u_t=u_{xx}, x\in (0,1), t\in (0,+\infty)\\
  u(x,0)=\varphi\\
  u_x(0,t)=u_x(1,t)=0
    \end{cases}
  \end{aligned}
  $$
  \par
  讨论$u$何时满足:$$
\lim_{t \to +\infty}u(x,t)=0, \forall x \in (0,1)
  $$
\end{problem}


\begin{solution}
  \par
分离变量法,设
$$
u(x,t)=X(x)T(t)
$$
\par
有
$$
\begin{aligned}
X(x)T'(t)=X''(x)T(t)\\
\frac{X''(x)}{X(x)}=\frac{T'(t)}{T(t)}=\lambda
\end{aligned}
$$
\par
$$
\begin{cases}
  X''(x)-\lambda X(x)=0\\
  T'(t)-\lambda T(t)=0
\end{cases}
$$
\par
同前讨论,$\lambda<0$,解两个ODE:

$$
\begin{cases}
  X(x)=C_1\cos (\sqrt{-\lambda}x)+C_2\sin (\sqrt{-\lambda}x)\\
  T(t)=C_3\mathrm{e}^{\lambda t}
\end{cases}
$$

$$
X_x(x)=-C_1\sqrt{-\lambda}\sin (\sqrt{-\lambda}x)+C_2\sqrt{-\lambda}\cos (\sqrt{-\lambda}x)
$$
\par
则
$$
\begin{cases}
C_2=0\\
-C_1\sqrt{-\lambda}\sin (\sqrt{-\lambda})+C_2\sqrt{-\lambda}\cos (\sqrt{-\lambda})=0
\end{cases}
$$
\par
于是,
$\lambda=-n^2\pi^2, n \in \mathbb{N}$.
\par
特解
$$
u_n=C_n\mathrm{e}^{-n^2\pi^2t}\cos(n\pi x)
$$
\par
通解
$$
u(x,t)=\sum_{n=1}^{+\infty}C_n\mathrm{e}^{-n^2\pi^2t}\cos(n\pi x)
$$
且
$$
\varphi(x)=\sum_{n=1}^{+\infty}C_n\cos (n\pi x)
$$
\par
于是,
$$
|C_n|=\left|2\int_{0}^{1}\varphi(x)\cos(n\pi x)\mathrm{d}x\right| \leqslant M, M>0,
\forall n \in \mathbb{N}
$$

\par
对无穷时间渐近:
$$
\begin{aligned}
|u(x,t)|
&=\left|\sum_{n=1}^{+\infty}C_n\mathrm{e}^{-n^2\pi^2t}\cos(n\pi x)\right|\\
&\leqslant M\sum_{n=1}^{+\infty}\mathrm{e}^{-n^2\pi^2t}\\
&\leqslant M\int_{0}^{+\infty}\mathrm{e}^{-\pi^2 tx^2}\mathrm{d}x\\
&=\frac{M}{2}\frac{1}{\sqrt{\pi t}}\to 0, t \to +\infty
\end{aligned}
$$
\par
于是,我们证明了
$$
\lim_{t \to +\infty}u(x,t)=0
$$
\par
并且这个极限是一致的.
\end{solution}





\end{document}