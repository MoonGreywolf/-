\documentclass[12pt, a4paper, oneside]{ctexart}
\usepackage{amsmath, amsthm, amssymb, bm, color, framed, graphicx, hyperref, mathrsfs}

\title{\textbf{梅加强——微分中值定理和拟微分中值定理}}
\author{张浩然}
\date{\today}
\linespread{1.5}
\definecolor{shadecolor}{RGB}{241, 241, 255}
\newcounter{problemname}
\newenvironment{problem}{\begin{shaded}\stepcounter{problemname}\par\noindent\textbf{题目\arabic{problemname}. }}{\end{shaded}\par}
\newenvironment{solution}{\par\noindent\textbf{解答. }}{\par}
\newenvironment{note}{\par\noindent\textbf{题目\arabic{problemname}的注记. }}{\par}

\begin{document}

\maketitle

\par
凸域就是开的凸集.

\begin{problem}
  设$f: D \to \mathbb{R}$在区域$D\subset \mathbb{R}^{n}$中处处可微.
  如果$f$的梯度场是常值的,证明$f$是线性函数.
\end{problem}


\begin{solution}
    \par
    设$f$的梯度场为$(u_1,u_2,\cdots, u_n)=u^{T}$,
    $\nabla \left(f(x)-u^{T}\cdot x\right)=u^T-u^T=0$
    \par
    任取$x,y \in D$, 根据微分中值定理:
    $$
    \begin{aligned}
    f(x)-u^T\cdot x-f(y)+u^T\cdot y&=\nabla\left(f(\xi)-u^{T}\cdot \xi\right)\cdot(x-y)\\
    &=0
    \end{aligned}
    $$
    \par
    $f(x)-u^T\cdot x=f(y)-u^T\cdot y $.
    \par
    存在$b \in \mathbb{R}$,使得$f(x)=u^T\cdot x +b$.
\end{solution}
\begin{note}
    \par
此处$b\neq 0$时,我们也说是线性函数,与高代要求可加性和齐次性不同,就像我们认为一元函数$y=2x+1$也是线性函数一样.
\end{note}


\begin{problem}
设$f:\mathbb{R}^n \to \mathbb{R}^n$处处可微. 当$x \in \mathbb{R}^n$时,矩阵$Jf(x)(Jf)^T(x)$的
最大特征值记为$\Lambda(x)$.
\par
若$\Lambda=\sup_{x \in \mathbb{R}^n}\Lambda(x)< +\infty$, 证明:
$$
\left\|f(x)-f(y)\right\|\leqslant \sqrt{\Lambda}\|x-y\|
$$
\end{problem}

\begin{solution}
    \par
     这是一个线性代数问题,熟知结论:$$
     \left|\lambda E_{n\times n}-Jf(x)(Jf)^T(x)\right|=\left|\lambda E_{n \times n}- (Jf)^T(x)Jf(x)\right|
     $$
     \par
     $Jf(x)(Jf)^T(x)$ 和$(Jf)^T(x)Jf(x)$具有相同特征值,最大特征值都为$\Lambda (x).$
    \par
    根据2-范数的拟微分中值定理:
    $$
    \|f(x)-f(y)\|_2\leqslant \|Jf(\xi)\|_2\cdot \|x-y\|_2
    $$
    \par
    范数是2-范数,那么$\|Jf(\xi)\|_2=\sqrt{\max \lambda((Jf)^T(\xi)Jf(\xi))}$, 
    则 $\|Jf(\xi)\|_2=\sqrt{\Lambda (\xi)}$
    $$
    \begin{aligned}
    \|f(x)-f(y)\|_2&\leqslant \|Jf(\xi)\|_2\cdot \|x-y\|_2\\
    &\leqslant \sqrt{\Lambda (\xi)}\cdot \|x-y\|_2\\
    &\leqslant \sqrt{\Lambda}\cdot \|x-y\|_2 \\
    \end{aligned}
    $$
    对于向量,2-范数和Frobenius范数相等,即$\|x\|=\|x\|_2, \forall x \in \mathbb{R}^n$
    $$
        \|f(x)-f(y)\|
        \leqslant \sqrt{\Lambda}\cdot \|x-y\| 
    $$

\end{solution}

\begin{note}

    \par
    这题也有点奇怪,此处范数应该只能是2-范数,不然我不会做……
    \par
    我们来证明2-范数的拟微分中值定理:$D\subset \mathbb{R}^n$是凸域, $f:D \to R^m$可微,任取
    $x,y \in D$,存在$\xi=\theta x +(1-\theta)y, \theta\in(0,1)$,使得:
    $$
    \|f(x)-f(y)\|_2\leqslant \|Jf(\xi)\|_2\cdot \|x-y\|_2
    $$
    \par
    证明:任取$x,y \in D$, $D\subset \mathbb{R}^n$是凸域,不妨设$f(x)\neq f(y)$,否则平凡.那么我们取$\sigma(t)=tx+(1-t)y, \forall t\in [0,1].$
    \par
    设函数$\varphi(t)=\left \langle f(x)-f(y), f(\sigma(t))\right \rangle$,这是$\mathbb{R}^m$上的标准内积.
    \par
    根据一元的微分中值定理:存在$\theta \in(0,1)$:
    $$
    \begin{aligned}
    &\varphi(1)-\varphi(0)=\varphi'(\theta)\\
    &\left \langle f(x)-f(y), f(x)-f(y) \right \rangle=\left \langle f(x)-f(y), Jf(\theta x+(1-\theta)y)(x-y)\right \rangle\\
    \end{aligned}
    $$
    \par
    我们利用$\|x\|=\|x\|_2, \forall x \in \mathbb{R}^m$,
    $$
    \begin{aligned}
    &\left \langle f(x)-f(y), f(x)-f(y) \right \rangle=\left \langle f(x)-f(y), Jf(\theta x+(1-\theta)y)(x-y)\right \rangle\\
    &\|f(x)-f(y)\|_2^2    =\left \langle f(x)-f(y), Jf(\theta x+(1-\theta)y)(x-y)\right \rangle\\
    &\|f(x)-f(y)\|_2^2    \leqslant \|f(x)-f(y)\|_2\cdot\| Jf(\theta x+(1-\theta)y)(x-y)\|_2\\
    &\|f(x)-f(y)\|_2   \leqslant \| Jf(\theta x+(1-\theta)y)(x-y)\|_2\\
    &\|f(x)-f(y)\|_2   \leqslant  \| Jf(\theta x+(1-\theta)y)\|_2\cdot\|(x-y)\|_2\\
\end{aligned}
    $$
    \par
    这里用到$\|Ax\|_2\leqslant \|A\|_2\|x\|_2$,于是定理成立.
    \par
    \quad\\
    \par
    我们叙述2-范数, 对于向量$x\in \mathbb{R}^n$,向量的2-范数定义如下:
    $$
    \|x\|_2=\sqrt{\sum_{i=1}^{n}x_i^2}
    $$
    根据2-范数诱导的矩阵范数称为矩阵2-范数,对于$m\times n$的实系数矩阵,定义如下:
   $$
   \|A\|_2=\sup_{\|x\|_2\neq 0}\dfrac{\|Ax\|_2}{\|x\|_2}=\sup_{\|x\|_2=1 }\|Ax\|_2
   $$
   \par
   于是,$\|A\|_2^2=\sup_{\|x\|=1}\|Ax\|_2^2=\sup_{\|x\|=1}(Ax)^TAx=\sup_{\|x\|=1}x^TA^TAx$.
   \par
   $A^TA$是实的对称矩阵,那么可以对角化,而且$\|Ax\|_2^2=x^TA^TAx\geqslant 0$, $A^TA$半正定,
   那么其有$n$个非负特征值,排列如下$\lambda_1\geqslant \lambda_2\geqslant \cdots \geqslant \lambda_n\geqslant 0$.
   那么存在正交矩阵$V$,使得:
   $$
   A^TA=V\begin{pmatrix}
    &\lambda_1 &\quad &\quad &\quad\\
    &\quad &\lambda_2 &\quad &\quad\\
    &\quad &\quad &\ddots &\quad\\
    &\quad &\quad &\quad &\lambda_n\\
   \end{pmatrix}
   V^T
   $$
   于是,
   $$
   A^TAV=V\begin{pmatrix}
    &\lambda_1 &\quad &\quad &\quad\\
    &\quad &\lambda_2 &\quad &\quad\\
    &\quad &\quad &\ddots &\quad\\
    &\quad &\quad &\quad &\lambda_n\\
   \end{pmatrix}
   $$
   设$V$的列向量为$b_1,b_2,\cdots,b_n$,它们都是单位向量(根据正交矩阵性质)它们为$A^TA$的特征向量,对应特征值$\lambda_1,\lambda_2,\cdots, \lambda_n$
   \par
   任取$i\neq j, b_i^Tb_j=0$(用到正交矩阵列向量正交) .
   于是,$b_1,b_2,\cdots,b_n$两两正交,构成$\mathbb{R}^n$上的一组标准正交基.
   \par
   于是设$x=k_1u_1+k_2u_2+\cdots+k_nu_n$,
   则$$
   \begin{aligned}
   x^TA^TAx&=(k_1u_1++k_2u_2+\cdots+k_nu_n)^T(k_1\lambda_1u_1+k_2\lambda_2u_2+\cdots+k_n\lambda_nu_n)\\
   &=k_1^2\lambda_1+\cdots+k_n^2\lambda_n\\
   &\leqslant \lambda_1(k_1^2+k_2^2+\cdots+k_n^2)\\
   &=\lambda_1\\
   &=\max \lambda(A^TA)
   \end{aligned}
   $$
   \par
   这是因为在$\|x\|_2=1, k_1^2+k_2^2+\cdots+k_n^2=1$条件下,只需要令$x=u_1$就可以取到等号,
   于是:
   $$
   \|A\|_2=\sup_{\|x\|_2\neq 0}\dfrac{\|Ax\|_2}{\|x\|_2}=\sup_{\|x\|_2=1 }\|Ax\|_2=\sqrt{\max \lambda(A^TA)}
   $$
\end{note}


\begin{problem}
设$D\subset \mathbb{R}^n$是开集, $f\in C^1(D)$. 如果$\overline{B_{r}(x^0)}\subset D$,则任给
$\varepsilon >0$, 存在$\delta >0$, 使得当$x,y \in \overline{B_{r}(x^0)}$且$\|x-y\|<\delta$时,
$$
\|f(y)-f(x)-Jf(x)(y-x)\|\leqslant \varepsilon\|x-y\|
$$
\end{problem}

\begin{solution}
   我们做最一般的证明,$f$是向量值函数时,我们用不了微分中值定理,但是可以用拟微分中值定理:
   $$
   g(z)=f(z)-Jf(x)z
   $$
满足在$\overline{B_{r}(x^0)}$上处处可微,对任意的$x,y\in\overline{B_{r}(x^0)}$那么存在$\theta\in (0,1)$:
$$
    \| g(y)-g(x)\|\leqslant  \|Jg(\theta x+(1-\theta)y)\|\|x-y\|
$$
也就是:
$$
\|f(y)-f(x)-Jf(x)(y-x)\|\leqslant\|Jf(\theta x+(1-\theta)y)-Jf(x)\|\|x-y\|
$$
根据$f\in C^1(D)$,任取$\varepsilon>0$,存在$\delta>0$, 使对于$x,y \in \overline{B_{r}(x^0)}$且$\|x-y\|<\delta$,则$\|Jf(y)-Jf(x)\|\leqslant \varepsilon$
,而又有:
$$
\|\theta x+(1-\theta)y-x\|=\|(1-\theta)(x-y)\|\leqslant \|x-y\|<\delta
$$
\par
使得:
$$
\|Jf(\theta x+(1-\theta)y)-Jf(x)\|\leqslant \varepsilon
$$
\par
进而,
$$
\begin{aligned}
\|f(y)-f(x)-Jf(x)(y-x)\|&\leqslant\|Jf(\theta x+(1-\theta)y)-Jf(x)\|\|x-y\|\\
&\leqslant \varepsilon\|x-y\|
\end{aligned}
$$
\end{solution}
\begin{note}
    \par
    我们终于证明了结论……
    \par
    至于为什么能在$\overline{B_r(x^0)}$上用拟微分中值定理,这时只需要认为它包含于一个处处可微的凸域$D'$($D'$包含于$D$)中就行了,根据欧式空间的性质,这样的凸域是存在的.
    \par
    
\end{note}

\begin{problem}
    设$D\subset \mathbb{R}^n$为凸域,$f\in C^{1,1}(D)$,即$f$处处可微,且存在常数$L$,使得
    $$
         \|Jf(x)-Jf(y)\|\leqslant L\|x-y\|, \forall x,y \in D.
    $$
    证明:
    $$
    \|f(y)-f(x)-Jf(x)(y-x)\|\leqslant \dfrac{1}{2}L\|x-y\|^2, \forall x,y \in D.
    $$
\end{problem}

\begin{solution}
    \par
    我们还是证明最一般的情况:$f$是$C^{1,1}(D)$的向量值函数.
    \par
    先推广定义,对一元向量值函数$G: [a,b] \to \mathbb{R}^m$,各分量为$g_1,g_2,\cdots,g_m$,
    其积分为:
    $$
    \int_{a}^{b}G(t)dt =\begin{pmatrix}
        \int_{a}^{b}g_1(t)\mathrm{d}t\\
        \int_{a}^{b}g_1(t)\mathrm{d}t\\
          \vdots\\
        \int_{a}^{b}g_1(t)\mathrm{d}t\\
    \end{pmatrix}
    $$
    $G$黎曼可积当且仅当各分量黎曼可积.
    \par
    对于多元向量值函数$f: D\subset \mathbb{R}^n \to \mathbb{R}^m$,$f_1,f_2,\cdots.f_m$为分量,都是多元函数,如果可微,
    那么$f_i(x+t(y-x))$在$t \in [0,1]$连续,如果$f_i'(x+t(y-x))$在$t\in[0,1]$黎曼可积,那么
    $$
    f_i(y)-f_i(x)=\int_{0}^{1}f_i'(x+t(y-x))(y-x)\mathrm{d}t
    $$
    于是我们可以写出:
    $$
    f(y)-f(x)=\int_{0}^{1}Jf(x+t(y-x))(y-x)\mathrm{d}t
    $$
    \par
    \quad
    \par
    引理:如果一元向量值函数$G:[a.b]\to \mathbb{R}^m$连续,那么
    $$
   \left \|\int_{a}^{b}G(t)\mathrm{d}t\right\|\leqslant \int_{a}^{b}\|G(t)\|\mathrm{d}t
    $$
    对于本题:
    $$
    \begin{aligned}
    \|f(y)-f(x)-Jf(x)(y-x)\|&=\left\|\int_{0}^{1}Jf(x+t(y-x))(y-x)\mathrm{d}t-Jf(x)(y-x)\right\|\\
    &=\left\|\int_{0}^{1}Jf(x+t(y-x))(y-x)\mathrm{d}t-\int_{0}^{1}Jf(x)(y-x)\mathrm{d}t\right\|\\
    &=\left\|\int_{0}^{1}[Jf(x+t(y-x))-Jf(x)](y-x)\mathrm{d}t\right\|\\
    &\leqslant \int_{0}^{1}\|[Jf(x+t(y-x))-Jf(x)](y-x)\|\mathrm{d}t\\
    &\leqslant \int_{0}^{1}\|Jf(x+t(y-x))-Jf(x)\|\cdot\|(y-x)\|\mathrm{d}t\\
    &\leqslant \int_{0}^{1}L\|t(y-x)\|\cdot\|(y-x)\|\mathrm{d}t\\
    &=L\|(y-x)\|^2\int_{0}^{1}t\mathrm{d}t\\
    &=\dfrac{1}{2}L\|x-y\|^2
    \end{aligned}
    $$
\end{solution}

\begin{note}
    \par
    引理:如果一元向量值函数$G:[a.b]\to \mathbb{R}^m$连续,那么
    $$
   \left \|\int_{a}^{b}G(t)\mathrm{d}t\right\|\leqslant \int_{a}^{b}\|G(t)\|\mathrm{d}t
    $$
    \par
    引理的证明: 范数是一个连续函数,那么$\|G(t)\|$黎曼可积, 任取$\varepsilon>0$,
    存在分划$a=t_0<t_1<\cdots<t_{p-1}<t_p=b$, 使得
    $$
    \left\|\int_{a}^{b}G(t)\mathrm{d}t-\sum_{1}^{p}G(t_i)(t_i-t_{i-1})\right\|<\varepsilon
    $$
    以及
    $$
    \left|\int_{a}^{b}\|G(t)\|\mathrm{d}t-\sum_{1}^{p}\|G(t_i)\|(t_i-t_{i-1})\right|<\varepsilon
    $$
    于是,
    $$
    \begin{aligned}
        \left\|\int_{a}^{b}G(t)\mathrm{d}t\right\|&\leqslant \left\|\sum_{1}^{p}G(t_i)(t_i-t_{i-1})\right\|+\varepsilon\\
        &\leqslant \sum_{1}^{p}\|G(t_i)\|(t_i-t_{i-1})+\varepsilon\\
        &\leqslant \int_{a}^{b}\|G(t)\|\mathrm{d}t+2\varepsilon
    \end{aligned}
    $$
    根据$\varepsilon$的任意性,
    $$
    \left \|\int_{a}^{b}G(t)\mathrm{d}t\right\|\leqslant \int_{a}^{b}\|G(t)\|\mathrm{d}t
     $$
\end{note}



\end{document}